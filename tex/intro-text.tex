\title{Numbers on Computers}

\section{Introduction \& tools}

The goals of Computational Physics are to learn the principles of
using computers to perform calculations in physics.  For example,
using computer simulations of the laws of physics enables us to
calculate (and therefore test) the predictions of the cosmological
theory of the formation of galaxies; one recent example is from the
\href{https://vimeo.com/93020014}{EAGLE simulations}.

The principles that underlie these techniques include how numbers are
represented, and the basic tools for calculating special functions,
random numbers, linear algebra, interpolation, root-finding and
optimization, differentiation, integration, and spectral analysis.

It is helpful to understand what we will not be doing:
\begin{itemize}
\item We will not here be considering explicitly problems in physics
  data analysis, but many of the techniques we learn here will be
  applicable in that context.
\item I will not be teaching you the principles of software
  engineering. To the extent possible, I will emphasize the importance
  of documentation, modularity, validation, and version control in
  writing stable, maintainable code. But do not underestimate the
  level of experience, discipline, and effort that good software
  engineering requires, and respect it when you see it.
\end{itemize}

I will lecture and demonstrate in class, and there will be homeworks
most of the semester. There will also be a set of large-ish
projects. You will work on these in pairs or triplets, and you will
let me know which teams you are on by Sept 21, at which point I will
assign topics.

Your recitations will be spent working on homeworks and projects, with
the TA available to help. The first recitation will be spent setting
up your laptops appropriately to perform the work for the course.

We must choose a language to work in. There are many available
computer languages. Useful ones are C, C++, Fortran, Java, Python, and
Julia. There are other field-specific languages in use as well
(e.g. in astronomy there are IRAF and IDL).

Here we will use Python, mostly because it is a modern language of
broad applicability, which also allows easy visualization within the
language itself. It has limitations: you would probably not want to
write a very computationally heavy simulation in Python, but more
likely directly in C++. But for many calculations it works fine and is
convenient, and in any case in this course we will concern ourselves
mostly with the concepts of computational physics rather than aiming
at the most efficient possible implementations.

The homeworks will be handed in in the form of Python scripts and
results. You will need to have an editor to edit your Python files and
a system to run them. Typically these days people use an Integrated
Development Environment (IDE), like Spyder or VSCode. Although I am
quite prescriptive about how you will hand in projects and homework, I
do not have strong opinions about what editors or IDEs you use.

I will show some examples in Jupyter Notebooks.  This is a
browser-based tool, which allows a similar interface to work on the
command line, but retains a clean record of what you did. It is good
for testing code, for exploratory work, and for presenting
results. \href{https://github.com/jakevdp/PythonDataScienceHandbook/tree/de0cc6bd317012d50ab3dd06e3cf4e256de1973f/notebooks}{The
Python Data Science Handbook} has a good introduction to their
use. But you should not get in the habit of using them for final
versions of research-grade software, which is why I do not accept
homework handed in as notebooks.

Unfortunately, for a class like this we have to go through the basics
of getting everybody set up on their computers and using the tools of
the course. So there is some boring stuff to get through before we get
to the fun stuff. 

First, let me note that there is a
\href{https://blanton144.github.io/computational-grad/}{\color{red} class
  website}, also found linked from NYU Brightspace. This will have
important resources including my class notes.

I want to describe what each of you will need in your environment. The
end point requirements are:
\begin{ditemize}
\item {\tt git}
\item {\tt \LaTeX}
\item {\tt Python}
\end{ditemize}
and as long as those work on your system you should be OK. In
principle, these should all be able to work on Windows, Linux, or Mac
OS X. Probably Linux and Mac OS X (which is Unix-based) are the
easiest though; honestly if you are a Physics graduate student (at
least in the current era) you really should have familiarity with and
access to the Unix operating system.

For {\tt git}, you will need to install the {\tt git} software and you
will also need to create a \href{http://github.com}{\color{red}
GitHub} account. You should create this account, and then start a
GitHub repository called {\tt phys-ga2000} specifically for this course
following the instructions on GitHub. The first problem set will
describe how problem sets will be handed in.

You will need to use \LaTeX\ for the homeworks and your class
report. On the class web site is a basic example of a report. You can
use this to get started if you are running \LaTeX\ on your laptop, and
sharing with your team through Github. The other option is to use
\href{http://overleaf.com}{\color{red} Overleaf}, which is a
cloud-based Latex document system, also good for sharing. If you use
Overleaf for your homeworks, you will upload the PDF to the problem
set directory.

To install Python and Jupyter Notebooks on your computer and enough of
its packages to function, I recommend the
\href{https://www.continuum.io/downloads}{\color{red} Anaconda Python
  3.11 (or later) distribution}. Other packages can be also be
installed using Anaconda's {\tt conda} utility or {\tt pip}. Key for
all science done with Python is the NumPy package.

I will illustrate on the projector how a Jupyter notebook works and
some of the basic features of Python.

\section{Python: an Object Oriented language}

Python is what is referred to as ``Object Oriented.'' There is a lot
of baggage associated with this classification, but the basic idea is
that entities in Python are not just variables and functions, but also
objects (in fact, the variables and functions are themselves
objects). 

I illustrate in the notebook how an object class is defined and
used. If you are used to procedural programming, object-oriented
programming requires some getting used to. You can use Python (almost)
purely procedurally, but using object classes gives you new ways to
make your code modular --- if you are careful. Also, NumPy and other
packages do force you to use classes. 

Some language is useful. In particular, objects have ``attributes''
and they have ``methods.'' Attributes are just quantities that are set
for an object instance. Methods are functions that the object instance
can run. 

\section{Representation of integers on computers}

A basic fact about computation that you need to understand regards how
numbers are represented on computers. All information stored on
computers is stored in units of ``bytes.''

\noindent{\bf What is a byte?}

\begin{answer}
  A byte is 8 bits. Each bit can be 0 or 1. It is like a Boolean:
  either 0 or 1, or False or True. You can express a byte as a series
  of eight bits, for example this one {\tt 01000110}.
\end{answer}

\noindent{\bf How can integers be represented by bytes? }

\begin{answer}
  The sequence of bits can be interpreted as a base-2 number. That is,
  the integer can be generated from the 8 bits as:
\begin{equation}
i = \sum_{i=0}^7 b_i \times 2^i
\end{equation}
E.g. assuming the ``lowest significance'' bit is on the right side of
01000110, this becomes:
\begin{equation}
i = 2^1 + 2^2 + 2^6 = 2 + 4 + 64 = 70
\end{equation}
If you store a number on a computer this way it is known as an
``unsigned 8-bit integer.''
\end{answer}

\noindent{\bf What is the highest integer that can be represented by
    a byte in this way?}

\begin{answer}
  It is set by the following formula using $n=8$:
  \begin{equation}
    i_{\rm max} = \sum_{i=0}^{n-1} 2^i = 2^n - 1
  \end{equation}
  which is $2^8 - 1 = 256 - 1 = 255$. If you try to increment an
  unsigned 8-bit integer beyond this limit, you will wrap back around to
  0.
\end{answer}

\noindent{\bf How might I represent negative numbers?}

\begin{answer}
  Usually this is done by reassigning the upper range (in this
  case 128\ldots 255) to the negative range (in this case $-128$\ldots
  -1). If this is done, then the data type is known as a ``signed
  8-bit integer.''

  There are two interesting things this does. First, because it is the
  upper half of the range, for negative numbers the highest
  significance bit is always set. Second, it means that if you are
  incrementing a signed 8-bit integer and go beyond 127, you will
  ``wrap around'' to -128.
\end{answer}

Most programming languages have a range of different integer types
that you may use. Specifically, 8-bit, 16-bit, 32-bit, and 64-bit are
the common ones (in addition to Boolean, or 1-bit).  These can be used
to represent increasingly large ranges of signed or unsigned integers.

32-bit is the usual standard integer type. However, an unsigned 32-bit
integer can only count up to about 4 billion --- this can actually be
a practical limit. 

\section{Representation of floating point numbers on computers}

Of course, for computation we need to be able to express
non-integers. Bits can be used to do so. The resulting numbers are
known as ``floating point'' numbers, e.g. in NumPy {\tt float32}
(single precision) or {\tt float64} (double precision), depending on
the number of bits. Please note that these definitions of single- and
double-precision are {\it mostly} standard these days but you may run
into older definitions at times.

Many books claim that all modern computational physics relies entirely
on 64-bit precision. Such statements reveal a limitation in the
authors' experience: there are plenty of contexts where 32-bit
precision is sufficient and 64-bit precision would be wasteful of
computing time, I/O time, or memory. You cannot escape the use of
judgment in determining whether or not to use high precision or not!
It will not matter much in this course however.

IEEE 754 defines standards for storing floating point numbers.  To
express a large enough dynamic range (more than just a factor of a few
billion for 32 bits), floating point numbers use something akin to
``scientific notation.'' Just like we write numbers like:
\begin{equation}
1.677 \times 10^{-34}
\end{equation}
to avoid the use of a lot of zeros every time we write them down, a
similar technique is used when expressing numbers on computers. In
particular, there are three types of bits. First, there is a ``sign''
bit ($\pm$). Second, there is mantissa, which is the analog of the
``1.677'' above. Third, there is the exponent, which is the analog of
the ``$-34$'' above.

More specifically, floating point numbers are broken up (e.g. for
32-bits) in the following way: bit 31 is the sign bit $s$; bits 30--23
express the 8-bit integer exponent $e$, and bits 22--0 are the
mantissa $f$. For a normal number, its value is:
\begin{equation}
\pm \left(1 + \sum_{i=0}^{22} f_i 2^{-(i-23)} \right) 2^{e-127}
\end{equation}

In reality, normal numbers are cases where $1\le e \le 254$. Special
cases are :
\begin{ditemize}
\item $e=0$, $f\ne 0$, so called ``subnormal'' numbers, where the
  ``1'' above is replaced with 0, and $e-127$ by $-126$.
\item $e=0$, $f=0$, a signed zero
\item $e=255$, $f=0$, a signed $\infty$
\item $e=255$, $f\ne 0$, ``NaN'' or ``Not a Number'' (e.g. the result
  of $0/0$).
\end{ditemize}

Inherent in any finite-bit representation on a computer is the effect
of truncation error in calculations. This is of enormous importance to
understand, because it affects the accuracy and stability of numerical
calculations. 

\noindent {\bf What is the range of numbers that can expressed in
  single precision?}

\begin{answer}
The highest number is about $2^{128}$ (i.e. with exponent of 127, and
the mantissa all 1s) or $3.4\times 10^{38}$. 

The lowest number is about $2^{-126 -23}$, i.e. a subnormal number
with $e=0$, with only the lowest significant bit set, or $1.4\times
10^{-45}$. 
\end{answer}

\noindent {\bf How does the computer add or subtract two floating
point numbers?}

\begin{answer}
Adding and subtracting integers is pretty straightforward.  To add
floating point numbers, it is just as simple {\it if} the floating
point numbers have the same exponent. In that case, the mantissa is
just added together, and if it rolls over beyond $2^{23}$ or below
$0$, the exponent is adjusted and the mantissa converted accordingly.

But if the two numbers have different exponents, first the exponent
has to be shifted. If the exponents are $e=100$ and $99$, then the
latter must be shifted one bit to the left up to 100, and the mantissa
shifted one bit to the right to account for the difference. However,
this new number is not necessarily precisely the same number that was
expressed before, because the lower significance bit of the number is
lost.

So in general the addition of the numbers might not yield precisely
the correct answer. This is known as {\it round-off error} and is a
major limitation to understand in computational work. 
\end{answer}

\noindent {\bf Imagine adding a small number ($10^{-8}$) to a larger
  one ($1$) on a computer. How does truncation error affect this?}

\begin{answer}
Broadly, it will mean a loss of precision in this addition; in this
specific case, you will lose all precision, because the result will be
exactly one at single precision!

This happens because when the computer adds these numbers, it first
needs to get them to the same exponent. Once that is true, it can just
add their mantissas. 

But if there is too large a dynamic range in the numbers it means that
the mantissa shifts so far to the right that it is completely zeroed
out. For single-precision, this 23 bits, or a relative amplitude
$2^{-23} \sim 10^{-7}$. In this case, adding a number smaller than
about this to $1$ will result in no change at all to one, which I will
demonstrate here!

Note that it is {\it relative} scale of the numbers that matters. As
long as you are further than $10^7$ from the edge the range of
single-precision numbers, it isn't the absolute scale of the numbers
that matters.

Finally, we're not going to be learning about exactly how arithmetic
is performed on computers, which is an entire curriculum on its own.
We just need to know this particular feature because it has such a
great effect on our calculations.
\end{answer}

\noindent {\bf If I have a large set of small numbers (e.g. 100
  million numbers all of order $10^{-8}$, how does this affect how the
  computer should best combine them to minimize truncation error?}
  
\begin{answer}
It should definitely not just start adding them up one by one. By the
time the sum has reached close to unity it will start losing precision
--- only a little bit per increment but it adds up. This is
catastrophic in this particular case for single precision.

Doing this at double precision is better behaved, but you will notice
that it still loses precision unnecessarily.

If I do this with a NumPy method {\tt sum()} to add up elements of an
array, I get back much better answers in both single- and
double-precision. The summation in this case is done in a more stable
manner apparently!
\end{answer}

\noindent {\bf Why shouldn't I just always use the highest precision
version of a number that I can?}
  
\begin{answer}
Whether to use 32-bit or 64-bit precision can be a subtle question. In
many simple applications, it does not matter and you can use
either. However, if you have strict precision requirements and/or
limited resources for your calculation it starts to matter.

Some calculations suffer from numerical instability. We'll get to
various examples later, but a common one is inverting a matrix that is
too close to being degenerate, which is more common than you might
think. In these cases, you'll want to use 64-bit precision. But why
not just always use the highest possible precision?

The first reason is that higher precision numbers take more memory. If
you need to keep lots of information in memory, the higher precision
will mean you can keep fewer numbers. For example, you may have a grid
of numbers describing, say, a density field; if you are
memory-limited, your grid will be lower resolution if you use higher
precision numbers.

The second reason is that higher precision numbers take longer to
operate on---in a multiplication, there is more for the computer if
there are more bits. If you are CPU-time limited, using higher
precision will mean you can take fewer time steps or have to use
larger time steps.

Sometimes, the approximation errors due to finite time steps and grid
spacing are larger than the 32-bit numerical precision errors. Thus,
using higher precision numbers can lead to {\it
lower} precision answers because of sacrifices you make in resolution
or number of time steps. At other times, numerical errors do matter
more. 
\end{answer}

\noindent {\bf Can you think of a way to multiply two numbers?}

The way that computers multiply numbers is more complicated than
addition obviously. There is a simple way to multiply integers, which
is basically long multiplication: you multiply each bit of the first
number by the bits of the second number; each multiplication produces
a shifted version of the second number; then you add them. But this is
fairly inefficient. There are improvements in this method that modern
CPU implement in hardware that prevent performing so many additions by
first storing all of the results from the pairwise bit
multiplications, and then recombining them cleverly, leading to a
scaling of $\log N_b$ in the number of bits $N_b$.

Multiplication of floating points is more complicated of course
because you need to add the exposures, worry about overflow etc.

\noindent {\bf What about dividing?}

Floating point (or integer) division is tricky numerically for the
same reasons it is tricky by hand. When computers see $a/b$, the usual
approach is to calculate $a\times(1/b)$, reducing the problem to
calculating $1/b$. 

To calculate $1/b$, the approach uses a root-finding approach called
Newton-Raphson that we will learn about later, and which it turns out
has a very simple form for this case. 

I hope I have impressed on you some of the complexity of even the
simplest calculations you have to do---entire courses are taught on
(and careers built on) the implementation of just what I cover
here. For this course, we will blithely move on, and take for granted
all of the hard work that computer scientists have done for physicists
and others.
