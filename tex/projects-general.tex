\documentclass[11pt, preprint]{aastex}

\usepackage{environ}
\usepackage{xcolor}
\usepackage{hyperref}
\usepackage{rotating}  
\usepackage{amsmath}

\newcommand{\todo}[1]{{\bf #1}}
\newcommand{\dd}[1]{\ensuremath{{\rm d}#1}}
\newcommand{\mat}[1]{\ensuremath{{\bf #1}}}
\DeclareMathOperator{\sech}{sech}

\newif\ifanswers

\NewEnviron{answer}{\ifanswers\color{blue}\expandafter\BODY\fi}

\newenvironment{ditemize}
{ \begin{list}{}{%
\setlength{\topsep}{0pt}% 
\setlength{\partopsep}{3pt}% 
\setlength{\itemsep}{1pt}\setlength{\parsep}{1pt}% 
\setlength{\itemindent}{0pt}\setlength{\listparindent}{12pt}%
\setlength{\leftmargin}{24pt}\setlength{\rightmargin}{0in}%
\setlength{\labelsep}{6pt}\setlength{\labelwidth}{6pt}%
\renewcommand{\makelabel}{\makebox[\labelwidth][l]{$\bullet$\hspace{\fill}}}}}
{\end{list}}


\begin{document}

\title{\bf Computational Physics Projects / General Instructions}

For Computational Physics, you have each been assigned a group and an
assignment. You should work together, and hand in the assignment as a
group.

The deliverables for this assignment are:
\begin{itemize}
\item Two intermediate drafts of the write-up (you will not be graded
  on content---you just have to hand in the draft), with deadlines in
  the syllabus. By the first draft you should definitely have some
  idea of how you are going to address the questions and have an
  outline. By the second draft, it would be best to have the major
  pieces of code you need written, even if all the analysis and plots
  are not done. I will give you feedback on these drafts that will be
  useful for your final project.
\item A final version of a write-up, in PDF formatted by \LaTeX. This
  should be 5--10 pages including figures. It should be written in
  clear, proofread, expository English, that explains the results and
  addresses all the tasks and questions in the project description. It
  should include figures demonstrating the results. It should include
  references when appropriate (use BibTeX to do the citation
  management).
\item A 15 minute presentation to be given to the class. This will
  occur in the week after classes end.
\item The software that produced the results you present, in a GitHub
  repositority. Your team should develop the code using this system,
  so you can easily coordinate. The software should be implemented as
  one or more Python scripts, which where appropriate imports modules
  that you may have created. Unless the project description specifies
  otherwise, use only standard Python modules (like {\tt os}, {\tt
    sys}, {\tt argparse}, etc), plus {\tt numpy}, {\tt scipy}, and
  {\tt matplotlib}; if there is another library you think you want to
  use, let me know. Each function and class should have a docstring
  with the inputs and outputs and other relevant information
  described, and other inline documentation as appropriate.
\end{itemize}

In general, the structure of the software should be:
\begin{itemize}
  \item One or more Python scripts that can be called from the command
    line (with perhaps command line arguments) to perform computation
    and output results as data files. These scripts should control the
    overall flow of the computation, but should make use of modules.
  \item One or more modules in separate files that contain the core
    computational functions and classes you are using, with
    well-defined inputs and outputs.
  \item One or more files that {\tt pytest} can be used on for unit
    tests.
  \item A separate script to create plots based on the data files.
\end{itemize}
Do not use a Jupyter notebook, and do not put the entire computation
and plotting in one file! Use the files to organize your work and to
make your software modular.

Some parts of the projects are experimental in the sense that I have
not implemented the code before myself, and I cannot guarantee that it
is possible to answer the question that I have asked. So you should
definitely ask me questions if there are things that seem
confusing/impossible.

The grading will be based on Completeness (i.e. all of the parts above
are complete), Correctness (i.e. they seem to be done correctly),
Carefulness (appropriate unit and functional tests have been done),
and Clarity (of the write-up and the code documentation).

\end{document}
