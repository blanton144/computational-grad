\documentclass[11pt, preprint]{aastex}

\include{computational_defs}

\begin{document}

\title{\bf Computational Physics / PHYS-GA 2000 / Problem Set \#3
\\ Due September 24, 2024}

You {\it must} label all axes of all plots, including giving the {\it
  units}!!

\begin{enumerate}
  \item Read Example 4.3 in Newman. Using successively larger matrices
    (10$\times$10, 30$\times$30, etc.) find empirically and plot how
    the matrix multiplication computation rises with matrix size. Does
    it rise as $N^3$ as predicted? Use both an explicit function
    (i.e. the one in the example) and use the {\tt dot()} method. How
    do they differ?

  \item Exercise 10.2 in Newman.

  \item Exercise 10.4 in Newman.

  \item Demonstrate that the central limit theorem works. Do so by
    generating random variate $y=N^{-1} \sum_{i=0}^N x_i$, where $x_i$
    is a random variate distributed as $\exp(-x)$ (you can use the
    {\tt np.random} library to generate the exponentially-distributed
    variates). First, calculate analytically how you expect the mean
    (that one should be easy) and variance of $y$ to vary with
    $N$. Show visually that for large $N$ the distribution of $y$
    tends towards Gaussian. Show as a function of $N$ how the mean,
    variance, skewness, and kurtosis of the distribution
    change. Estimate at which $N$ the skewness and kurtosis have
    reached about 1\% of their value for $N=1$.

\end{enumerate}

\end{document}
