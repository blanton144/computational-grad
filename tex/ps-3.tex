\documentclass[11pt, preprint]{aastex}

\usepackage{environ}
\usepackage{xcolor}
\usepackage{hyperref}
\usepackage{rotating}  
\usepackage{amsmath}

\newcommand{\todo}[1]{{\bf #1}}
\newcommand{\dd}[1]{\ensuremath{{\rm d}#1}}
\newcommand{\mat}[1]{\ensuremath{{\bf #1}}}
\DeclareMathOperator{\sech}{sech}

\newif\ifanswers

\NewEnviron{answer}{\ifanswers\color{blue}\expandafter\BODY\fi}

\newenvironment{ditemize}
{ \begin{list}{}{%
\setlength{\topsep}{0pt}% 
\setlength{\partopsep}{3pt}% 
\setlength{\itemsep}{1pt}\setlength{\parsep}{1pt}% 
\setlength{\itemindent}{0pt}\setlength{\listparindent}{12pt}%
\setlength{\leftmargin}{24pt}\setlength{\rightmargin}{0in}%
\setlength{\labelsep}{6pt}\setlength{\labelwidth}{6pt}%
\renewcommand{\makelabel}{\makebox[\labelwidth][l]{$\bullet$\hspace{\fill}}}}}
{\end{list}}


\begin{document}

\title{\bf Computational Physics / PHYS-UA 210 / Problem Set \#3
\\ Due September 20, 2019}

You {\it must} label all axes of all plots, including giving the {\it
  units}!!

\begin{enumerate}
  \item Exercise 4.3 of Newman.

  \item Read Example 4.3 in Newman. Using successively larger matrices
    (10$\times$10, 30$\times$30, etc.) find empirically and plot how
    the matrix multiplication computation rises with matrix size. Does
    it rise as $N^3$ as predicted? Use both an explicit function
    (i.e. the one in the example) and use the {\tt dot()} method. How
    do they differ?

  \item Using the functions in the {\tt numpy.random} module, generate
    an ensemble of 10,000 one-dimensional random walks, each of 1,000
    steps, where each step is determined by the normal distribution
    (Gaussian with a standard deviation of unity). Show using plots
    how the mean and standard deviation of the distribution of walker
    positions grows with the number of steps. Can you explain the
    dependence of the standard deviation on the number of steps?

\end{enumerate}

\end{document}
