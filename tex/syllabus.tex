\documentclass[11pt, preprint]{aastex}
\usepackage{hyperref} 
\usepackage{rotating}
 
\setlength{\footnotesep}{9.6pt}

\newcounter{thefigs}
\newcommand{\fignum}{\arabic{thefigs}}

\newcounter{thetabs}
\newcommand{\tabnum}{\arabic{thetabs}}

\newcounter{address}

\begin{document}

\title{\bf Computational Physics / PHYS-GA 2000}
~
~

% Should add:
%  - Romberg integration
%  - Better intro to symplectic methods 
%  - Multi-D root-finding
%  - Multigrid methods

% Special topics
% - Interpolation
% - Profiling code

\noindent This course teaches computational physics for physics PhD
students. Classes meet Tuesday and Thursday 9:30am to 10:45am, in Room
208 of 194 Mercer.

\noindent The textbook is {\it Computational Physics}, by Mark
Newman. I will draw material also from {\it Computational Physics}, by
Philipp Scherer. Not a lot of creativity in book titles in this
field. Other resources are:
\begin{itemize}
\item
  \href{https://jakevdp.github.io/PythonDataScienceHandbook/}{Python
    Data Science Handbook (PDSH)} by Jake Van Der Plas.
\item {\it Numerical Recipes}, by Press, Teukolsky, Vetterling, and
  Flannery. This book is mostly useful for its insight and not for its
  code.
\end{itemize}

\noindent If you have never programmed in Python before then Chapter 2
of the book will require your special attention. There are also many
online resources for learning the basics of Python. I can recommend
Software Carpentry and also Prof.~David Pine's {\it Introduction to
  Python for science and engineering}, available online through Bobst
Library.

\noindent Prof. Blanton's office is Room 941 of 726 Broadway, and his
email is {\tt blanton@nyu.edu}. You can come to ask questions about
computational physics (or any other subject!) on Tuesdays 11:00am to
12:15pm, or by appointment.

\noindent The teaching assistant is Valentino Foit ({\tt
  foit@nyu.edu}) Recitation is Friday, 11am--12:15pm, in Room 1045
of 726 Broadway. This time will primarily consist of working on
homework assignments.

\noindent The class will be participatory. Please read the assignments
          {\it before} attending class; you will be expected at
          certain points to follow along with calculations on your
          computer.

\noindent There will be no exams in this course, but there will be a
pretty heavy load of assignments:
\begin{itemize}
\item {\it Weekly homeworks}: You may consult with each other about
  the homeworks, but you must write your own individual code and
  report.
\item {\it Semester project}: Performed in groups of two or three
  students each. I have deadlines for two intermediate drafts of this
  project; the projects are designed such that you will be able to
  complete them in stages over the semester based on material
  previously covered in class. The project culminates in a written
  report and a presentation in December.
\end{itemize}
All material handed in will consist of reports written in \LaTeX\ (the
physics standard typesetting system) and as documented Python code
which the TA and professor will be able to run to produce the data and
plots. You will receive rather specific templates and instructions
about code standards to follow!

\noindent It is common now for many people to make use of generative
AI tools (e.g. Co-Pilot) for programming or other tasks. {\it If you
  do so in the homeworks or the project, I require you to document
  doing so, providing the prompt and the raw output.} More generally,
just remember that these tools are only useful if you can determine
that they are working correctly, so using them doesn't save you from
checking the results. I do not think there is any reliable evidence
right now that they save significant time.

\noindent Grades are based on problem sets (65\%), the large project
and presentation (25\%), and class participation (10\%).

\noindent The classes will proceed as follows (subject to
revision!). 

\baselineskip 0pt
\begin{table}[h!]
\footnotesize
\begin{tabular}{|c|c|c|c|}
\hline
{\it Date} & {\it Topic} & {\it Reading} & {\it Problem Sets} \cr  
\hline 
2024-09-03 (T) & Numbers on computers  & Ch.~1, 2, 3 & \cr
2024-09-05 (R) & Arrays \& Numerics  & PDSH, Ch. 4 \cr
2024-09-10 (T) & Random Numbers     & Ch.~10.1--10.2 & {\bf PS\#1} \cr
2024-09-12 (R) & Interpolation      & Ch.~5.1--5.2 & \cr
2024-09-17 (T) & Integration        & Ch.~5.3--5.4 & {\bf PS\#2} \cr
2024-09-19 (R) & Integration        & Ch.~5.5--5.9 & Teams Determined \cr
2024-09-24 (T) & Differentiation    & Ch.~5.10--5.11 & {\bf PS\#3}\cr
2024-09-26 (R) & Differentiation    & & \cr
2024-10-01 (T) & Linear Algebra     & Ch.~6.1 & {\bf PS\#4}\cr
2024-10-03 (R) & Linear Algebra     & Ch.~6.1 & \cr
2024-10-08 (T) & Eigensystems & & {\bf PS\#5} \cr
2024-10-10 (R) & Eigensystems       & Ch.~6.2 & \cr
2024-10-15 (T) & {\bf Legislative Monday---no class}       & Ch.~6.2 & {\bf PS\#6} \cr
2024-10-17 (R) & Root-finding       & Ch.~6.3 & \cr
2024-10-22 (T) & Minimization       & Ch.~6.4 & {\bf Project draft \#1
  due}\cr
2024-10-24 (R) & Minimization       & Ch.~6.4 & \cr
2024-10-29 (T) & Minimization       & Ch.~6.4 & {\bf PS\#7} \cr
2024-10-31 (R) & Fourier Analysis     & Ch.~7 & \cr
2024-11-05 (T) & Ordinary DEs       & Ch.~8.1--8.3 & {\bf PS\#8} \cr
2024-11-07 (R) & Ordinary DEs       & Ch.~8.4--8.5 &  \cr
2024-11-12 (T) & Ordinary DEs       & Ch.~8.6 & \cr
2024-11-14 (R) & Partial DEs       & Ch.~9.1 & \cr
2024-11-19 (T) & Partial DEs        & Ch.~9.2 & {\bf Project draft \#2
  due}\cr
2024-11-21 (R) & {\bf Thanksgiving, no class} & & \cr
2024-11-26 (T) & Partial DEs        & Ch.~9.3 & {\bf PS\#9} \cr
2024-11-28 (R) & Partial DEs        & Ch.~9.3 & \cr
2024-12-03 (T) & Markov Chain Monte Carlo & --- & \cr
2024-12-05 (R) & Gaussian Processes & --- & \cr
2024-12-10 (T) & Gaussian Processes & --- & {\bf PS\#10} \cr
2024-12-12 (R) & Using GPUs         & --- & Final project due\cr
{\it Exam week} & Project presentations & --- & \cr
\hline
\end{tabular}
\end{table}

\end{document}

