These notes draw heavily on {\it Numerical Recipes}, a valuable
resource for entry-level understanding of numerical methods relevant
to physics.

Partial differential equations (PDEs) are distinguished from ordinary
differential equations by the presence of more than one independent
variable, which may be time or space. Generally speaking, they are
applicable to continuous systems: fluids, electromagnetic fields,
gravity, general relativity, etc. 

There are several types of PDEs, which can be classified as follows:
\begin{itemize}
\item Hyperbolic: second-order derivatives, with opposite signs, like
  the wave equation:
  \begin{equation}
\frac{\partial^2 u}{\partial t^2} - v^2 \frac{\partial^2 u}{\partial
  x^2} = 0
  \end{equation}
\item Elliptic: second-order derivatives, with like signs, like the
  Laplace equation:
  \begin{equation}
\frac{\partial^2 u}{\partial x^2} + \frac{\partial^2 u}{\partial
  y^2} = 0
  \end{equation}
\item Parabolic: first-order and second-order mixed, like the
  diffusion equation:
  \begin{equation}
\frac{\partial u}{\partial t} - \frac{\partial}{\partial
  x}\left(D\frac{\partial u}{\partial x}\right) = 0
  \end{equation}
\end{itemize}

These sorts of equations are more complex than ODEs, mainly because
you cannot reduce the equations to a single derivative of the state
variable with the single independent variable. Furthermore, you cannot
as easily set the boundary conditions; at least, there are many more
ways to. 

The simplest way to see this is the case of the Laplace equation. To
solve Laplace for an electric potential within some region requires
setting the potential at the surface of that region (or its
derivative, or some combination). Clearly this is a bit
complicated. Also, it is not at all obvious whether a solution with
the desired boundary conditions is necessarily allowed by the
equations (with Laplace, this is not really a problem of course, but
it can be with more general equations).

The setting of the boundary conditions is an extremely important
feature of any PDE problem, and usually defines the basic watersheds
between numerical approaches. For example, you can imagine setting
$u(x=0, t)$ and using the wave or diffusion equations to integrate in
time.  These are {\it Cauchy} or {\it initial value} problems.
But for Laplace, you need to specify {\it boundary values}, either of
$u$ ({\it Dirichlet}) or its derivative ({\it Neumann}).

\section{Initial value problems}

We will concern ourselves first with initial value problems. We will
first consider the {\it flux-conservative initial value
  problems}. These are of the form:
\begin{equation}
  \frac{\partial \vec{q}}{\partial t} =
  - \frac{\partial \vec{F}\left(\vec{q}\right)}{\partial x}
\end{equation}
$\vec{F}$ is called the {\it conserved flux}.

For example, take the wave equation. It can be rewritten as:
\begin{eqnarray}
\frac{\partial r}{\partial t} &=&  v \frac{\partial s}{\partial x} \cr
\frac{\partial s}{\partial t} &=&  v \frac{\partial r}{\partial x}
\end{eqnarray}
where:
\begin{eqnarray}
r &=& v \frac{\partial q}{\partial x}\cr
s &=& \frac{\partial q}{\partial t}
\end{eqnarray}

In this case, we can write:
\begin{equation}
\vec{F}\left(\vec{q}\right) = 
\left(\begin{array}{cc}
  0 & -v \cr
  -v & 0 
\end{array}\right) \cdot \vec{q}
\end{equation}

However, we will start with an even simpler example:
\begin{equation}
\frac{\partial q}{\partial t} = - v\frac{\partial q}{\partial x}
\end{equation}

\noindent {\bf What is the general solution of this problem given
  some initial conditions $q_0(x)$?}

\begin{answer}
  You can guess the answer by considering:
  \begin{equation}
    {\Delta q} \approx (-v \Delta t) \frac{\partial q}{\partial x}
  \end{equation}
  $-v\Delta t$ represents a distance. So the change in $q$ is just the
  gradient $q$ times this distance. That is, the new $q(x)$ is equal
  to the old $q(x-v\Delta t)$. 

  Then it becomes clear that the solution of this equation will be
  $q(x) = q_0(x-vt)$ --- that is, the equations just propagate the
  initial values forward at speed $v$.

  This is the simplest example of an {\it advective} equation. 
\end{answer}

\section{A simple method that doesn't work}

We will be looking at solving PDEs with finite difference related
methods. These are not the only methods, but they are a good place to
start.

To perform the finite differencing, we will take equal spaces in time
and space:
\begin{eqnarray}
x_j &=& x_0 + j \Delta x \cr
t_n &=& t_0 + n \Delta t
\end{eqnarray}

We can approximate the equation as:
\begin{equation}
  \frac{q_j^{n+1} - q_j^{n}}{\Delta t} = - v \left(
  \frac{q_{j+1}^n - q_{j-1}^n}{2\Delta x}
  \right)
\end{equation}
Then, given the initial conditions at $n=0$ and all $j$, we can
generate $q$ at $n=1$ explicitly from the current time step. 

In the notebook, I implement this very simple method. It clearly
doesn't work very well!!

\section{Stability Analysis}

To see why this happens, we need to perform a stability analysis. To
do so using the {\it von Neumann stability analysis}, we will search
for solutions of linearized equations under constant
coefficients. For the advective equation, the right hand side is
linear already, so we do not need to linearize it. 

\noindent {\bf Explain to me what ``linearizing'' a nonlinear
  expression is?}

\begin{answer}
It means to take small perturbations around a solution, and Taylor
expand the expression, keeping only the linear terms.  E.g.:
\begin{equation}
\vec{F}\left(\vec{q}\right) \approx \vec{F}\left(\vec{q}_0\right) +
\frac{\partial \vec{F}}{\partial \vec{q}} \cdot \delta\vec{q}
\end{equation}
\end{answer}

In any case, under the conditions of linearity and constant
coefficients, the solutions to the finite difference equation have the
form:
\begin{equation}
q_j^n = \xi^n \exp\left(i k j \Delta x\right)
\end{equation}
with some choice for $k$. 

\noindent {\bf Under what conditions is this mode unstably growing in
  time?}

\begin{answer}
When $\xi>1$, then this grows exponentially with $n$! If $\xi < 1$, it
dies away. Note that you should think of this analysis as looking at
what happens to a perturbation away from the correct solution.
\end{answer}

For the difference question above, if you plug this in:
\begin{eqnarray}
  \frac{q_j^{n+1} - q_j^{n}}{\Delta t} &=& - v \left(
  \frac{q_{j+1}^n - q_{j-1}^n}{2\Delta x} \right) \cr
  \xi^n (\xi - 1) \exp\left( i k j \Delta x\right) &=&
  - \frac{v \Delta t}{2\Delta x} \xi^n \left(
  \exp\left(i k (j+1) \Delta x\right)  - 
  \exp\left(i k (j-1) \Delta x\right) \right) \cr
  (\xi - 1) &=&
  - \frac{v \Delta t}{2\Delta x} \left(
  \exp\left(i k \Delta x\right)  - 
  \exp\left(- i k \Delta x\right) \right) \cr
  \xi &=&
  1 - i \frac{v \Delta t}{\Delta x} 
  \sin\left(k \Delta x\right)
\end{eqnarray}

\noindent {\bf What does this result mean for the stability of this
  method? Is it stable for any scale of fluctuation?}

\begin{answer}
The magnitude of $|\xi|>1$ for all values of $k$, so perturbations
around the solution at all scales grow exponentially! This is an
unconditionally bad method.
\end{answer}

\section{Lax Method}

The Lax method is a stable alternative, which just reads:
\begin{equation}
  \frac{q_j^{n+1} - \frac{1}{2}\left(q_{j+1}^{n} +
    q_{j-1}^{n}\right)}{\Delta t} = - v \left(
  \frac{q_{j+1}^n - q_{j-1}^n}{2\Delta x}
  \right)
\end{equation}

\noindent {\bf Describe the difference between the Lax method and the
  original method in words.}

\begin{answer}
This just replaces the estimate of $q_j$ at step $n$ with the average
of the neighboring points. 
\end{answer}

We can evaluate the stability of this method as well: 
\begin{eqnarray}
  \frac{q_j^{n+1} - \frac{1}{2}\left(q_{j+1}^{n} +
    q_{j-1}^{n}\right)}{\Delta t} &=& - v \left(
  \frac{q_{j+1}^n - q_{j-1}^n}{2\Delta x} \right) \cr
  \xi^n \left(\xi \exp(i k j \Delta x) - 
  \frac{1}{2} \left(\exp(i k (j+1) \Delta x\right) +
  \exp\left(i k (j-1) \Delta x \right) \right)
  &=& - i \frac{v \Delta t}{\Delta x} 
  \sin\left(k \Delta x\right) \xi^n \exp(ikj \Delta x) \cr
  \xi - \cos(k \Delta x) 
  &=& - i \frac{v \Delta t}{\Delta x} 
  \sin\left(k \Delta x\right) 
\end{eqnarray}
and thus:
\begin{equation}
  \xi = \cos(k \Delta x) 
   - i \frac{v \Delta t}{\Delta x} 
  \sin\left(k \Delta x\right) 
\end{equation}

\noindent {\bf What does the stability requirement mean in this case?}

\begin{answer}
We evaluate the amplitude of $\xi$:
\begin{eqnarray}
\left|\xi\right|^2 &=& \cos^2 k\Delta x + \left(\frac{v \Delta t}{\Delta
  x}\right)^2 \sin^2 k \Delta x \cr
&=& \cos^2 k\Delta x + \sin^2 k \Delta x  + 
\left[\left(\frac{v \Delta t}{\Delta
  x}\right)^2 - 1\right] \sin^2 k \Delta x \cr
&=& 1 +
\left[\left(\frac{v \Delta t}{\Delta
  x}\right)^2 - 1\right] \sin^2 k \Delta x < 1
\end{eqnarray}
or:
\begin{equation}
\label{eq:courant1}
\left[\left(\frac{v \Delta t}{\Delta
  x}\right)^2 - 1\right] \sin^2 k \Delta x < 0
\end{equation}
which requires:
\begin{equation}
\left(\frac{v \Delta t}{\Delta x}\right)^2 < 1
\end{equation}

This means that the time step you take is related to the resolution
you are trying to recover by:
\begin{equation}
\frac{\Delta x}{\Delta t} < v
\end{equation}
\end{answer}

\noindent {\bf Can you think of a heuristic interpretation of this
  requirement?} 

\begin{answer}
The time step needs to be shorter than the time it will take the
signal in the system to propagate across a single cell. This
requirement is the {\it Courant condition}. This basic idea reappears
in many different problems.
\end{answer}

Things are actually a bit trickier, as we will see in the notebook. If
I set $\Delta t = \Delta x / v$, we get exactly the right behavior.
This is because in that case $\xi=1$ exactly and so there is no
increase --- or decrease --- in the signal. But, if the time step is a
bit too large, we see the signal get unstable. If the time step is a
bit too small, we see the signal damp away! We had better understand
what is going on, because in real applications, where we don't know
exactly what we should expect, we can easily be tricked.

Note that if we take a very broad Gaussian, it is less affected. What
is going on here is that the small wavenumber $k$, large wavelength
solutions are closer to actually being equal to zero in
(\ref{eq:courant1}), and this leads to $\xi=1$. So when the time step
is too small, it tends to damp the high wavenumber, short wavelength
modes, but not the longer ones. This causes the Gaussian to broaden
(and also affects the fractional width of the narrow Gaussian more
quickly than the broad one). 

We can look directly at some sinusoidal waves to see this at work a
little more cleanly, which the notebook does.

\section{Numerical Viscosity}

We can rewrite the numerical equation for Lax as follows:
\begin{eqnarray}
  \frac{q_j^{n+1} - \frac{1}{2}\left(q_{j+1}^{n} +
    q_{j-1}^{n}\right)}{\Delta t} &=& - v \left(
  \frac{q_{j+1}^n - q_{j-1}^n}{2\Delta x}
  \right) \cr
  \frac{q_j^{n+1} - q_j^n}{\Delta t}
  &=& 
    - \frac{1}{2}\frac{\left(q_{j+1}^{n} - 2 q_j^{n} +
    q_{j-1}^{n}\right)}{\Delta t}
      - v \left(
  \frac{q_{j+1}^n - q_{j-1}^n}{2\Delta x}
  \right) 
\end{eqnarray}
This is an FTCS-type equation for:
\begin{equation}
  \frac{\partial q}{\partial t} = - \frac{\partial q}{\partial x}
  + \frac{\Delta x^2}{2\Delta t} \frac{\partial^2 q}{\partial x^2}
\end{equation}
The second term here is like a viscosity. So although the advective
equation itself has no viscosity, the numerical equations do have
viscosity. This damps perturbations on small scales, like with regular
viscosity. 

Whether the viscosity matters or not is a practical question. It
affects all scales at some rate, and the question is whether you are
looking at sufficiently large scales that it does not matter.

\section{Leapfrog Time Stepping}

The Lax method is second order in space but first order in time. If we
can store two time steps, we can achieve higher order in time. The
basic idea is to calculate the space derivatives every step, and the
time derivatives every other step. 

In general this yields:
\begin{equation}
q_j^{n+1} - q_j^{n-1} = - \frac{\Delta t}{\Delta x} \left(F_{j+1}^n -
F_{j-1}^n\right)
\end{equation}

For the advective equation:
\begin{equation}
q_j^{n+1} - q_j^{n-1} = - \frac{ v \Delta t}{\Delta x} \left(q_{j+1}^n
- q_{j-1}^n\right)
\end{equation}

If we evaluate this under the von Neumann stability analysis we find:
\begin{eqnarray}
\xi^2 - 1 &=& - \frac{v \Delta t}{\Delta x} \xi \left(\exp(i k \Delta x)
- \exp(- i k \Delta x) \right) \cr
\xi^2 - 1 &=& - 2i \xi \frac{v \Delta t}{\Delta x} \sin k\Delta x 
\end{eqnarray}
which then yields:
\begin{equation}
\xi = - i \frac{v \Delta t}{\Delta x} \sin k\Delta x \pm
\sqrt{1 - \left(\frac{v \Delta t}{\Delta x} \sin k\Delta x \right)^2}
\end{equation}
and so as long as the square root is of a real number:
\begin{equation}
|\xi|^2 = \left(\frac{v \Delta t}{\Delta x} \sin k\Delta x \right)^2 +
 1 - \left(\frac{v \Delta t}{\Delta x} \sin k\Delta x \right)^2 = 1
\end{equation}
Then the conditions of stability for all $k$ become:
\begin{equation}
\Delta t < \frac{\Delta x}{v}
\end{equation}
i.e. the Courant condition again.

\section{Diffusion Equation}

Another typical type of equation is the diffusion equation, generally
of the form:
\begin{equation}
\frac{\partial q}{\partial t} = \frac{\partial}{\partial
x} \left(D \frac{\partial q}{\partial x}\right)
\end{equation}
This too can be put in flux-conservative form where $F = -D \partial
u/\partial x$. 

Note you have actually already solved this problem in your homework,
and for the simple, linear case we are discussing of course that is
the preferred method.

Before actually discussing a specific method, it is worth asking what
quantity is going to be relevant to the time step.

\noindent {\bf From dimensional analysis, how can you construct a
quantity with units of time from the diffusion coefficient $D$ and a
chosen spatial grid size $\Delta x$?}

\begin{answer}
$D$ has units of length$^2$ time$^{-1}$. So $\Delta x^2 / D$ is a
quantity of units of time. It is the diffusion time across a cell. 
\end{answer}

We will define a unitless quantity:
\begin{equation}
\alpha = \frac{D \Delta t}{\Delta x^2}
\end{equation}

We can perform the diffusion equation finite differencing with:
\begin{equation}
\frac{q_j^{n+1} - q_j^{n}}{\Delta t} = D \left[\frac{q_{j+1}^n - 2
q_j^n + q_{j-1}^n}{\Delta x^2}\right]
\end{equation}
which can be rearranged to:
\begin{equation}
q_j^{n+1} = q_j^{n} + \alpha \left[q_{j+1}^n - 2
q_j^n + q_{j-1}^n\right]
\end{equation}

Performing a stability analysis on this yields:
\begin{eqnarray}
\xi &=& 1 + \alpha \left[e^{ik\Delta x} - 2 + e^{-ik \Delta
x}\right] \cr
&=& 1 + 2 \alpha \left[\cos(k\Delta x) - 1 \right] \cr
&=& 1 + 2 \alpha \left[1 - 2\sin^2\left(\frac{k\Delta x}{2}\right) - 1 \right] \cr
&=& 1 - 4 \alpha \sin^2\left(\frac{k\Delta x}{2}\right) 
\end{eqnarray}

\noindent {\bf What does this make the condition on stability?}

\begin{answer}
This method remains stable ($|\xi|<1$) for all $k$ as long as $\alpha
< 1/2$.
\end{answer}

The problem in this case is that we are most likely interested in the
diffusion on much larger scales than a single grid cell. If we are
interested in some scale $\lambda$, then the number of time steps it
will take to reach that scale is $N_T \sim (\lambda / \Delta
x)^2$. That's fine for the diffusion equation in its simplest form:
just take $\Delta x$ to be comparable to the scales you care
about. But if you are doing a problem numerically, usually it won't be
a pure diffusion problem, and then there are other features in the
problem that need to be tracked at higher resolution. So we want a way
to retain the spatial resolution but take longer time steps.

We can take larger time steps if we are willing to sacrifice accuracy
at small scales. One version of this to construct an {\it implicit}
differencing method:
\begin{equation}
\frac{q_j^{n+1} - q_j^{n}}{\Delta t} = D \left[\frac{q_{j+1}^{n+1} - 2
q_j^{n+1} + q_{j-1}^{n+1}}{\Delta x^2}\right]
\end{equation}
Here the right-hand side uses the future time step values!

\noindent {\bf How do we solve for $q^{n+1}$ in this case?}

\begin{answer}
This is still a linear system of equations. It has $N$ unknowns
($q_j^{n+1}$), and you have $N$ knowns ($q_j^{n}$). So it can be solved
as a sparse matrix problem. 

Specifically:
\begin{equation}
-\alpha q_{j-1}^{n+1} + (1 + 2\alpha) q_j^{n+1} - \alpha q_{j+1}^{n+1}
 = q_j^n
\end{equation}
and you need to set boundary conditions at the edge.

This is a tridiagonal system that you can solve easily with sparse
matrix techniques. I.e. backsubstituting a tridiagonal system is not
very hard.
\end{answer}

We can check the stability of the system:
\begin{equation}
\xi - 1 = \alpha \xi \left(e^{ik\Delta x} - 2 + 2 e^{ik\Delta x}\right)
\end{equation}
and using the same trigonometry as before:
\begin{equation}
\xi \left[ 1+ 4 \alpha \sin^2\left(\frac{k\Delta x}{2}\right)\right] =
1
\end{equation}
and so:
\begin{equation}
\xi = \frac{1}{ 1+ 4 \alpha \sin^2\left(\frac{k\Delta x}{2}\right)}
\end{equation}
This means that this is always stable. You can see how these implicit
methods can end up being very stable, because you'll end up with terms
in the denominator.

But it can't possibly be accurate at small scales for $\alpha>1$). It
tends to drive the solution to the solution of:
\begin{equation}
\frac{\partial^2 q}{\partial x^2} = 0
\end{equation}
which you can see by looking at the matrix equation above and setting
$\alpha\gg 1$ --- the ``source'' term goes to zero and you are left
with the equilibrium solution.  That's not a crazy result, but it does
mean that if there is structure on small scales in your initial
conditions, this dissipates, and perhaps not accurately. 

We can achieve higher accuracy by using a second-order-in-time
technique.  This is the most commonly used technique, and is called
the {\it Crank-Nicolson} method. It is simply the average of the two
above methods. 
\begin{equation}
\frac{q_j^{n+1} - q_j^n}{\Delta t}
= \frac{\alpha}{2}
\left[
  \left(q_{j+1}^{n+1} - 2 q_j^{n+1} + q_{j-1}^{n+1}\right) + 
  \left(q_{j+1}^{n} - 2 q_j^{n} + q_{j-1}^{n}\right)
\right]
\end{equation}
This is second order in time because each side is centered on $n+1/2$
and therefore over the interval $n$ to $n+1$ the first order error
terms integrate out. If you work it out, it is also unconditionally
stable, with:
\begin{equation}
 \xi = \frac{1 - 2 \alpha \sin^2\left(\frac{k\Delta x}{2} \right)}{1 + 2 \alpha \sin^2\left(\frac{k\Delta x}{2} \right)}
\end{equation}

How do we deal with generalizations of the diffusion equation, e.g. if
$D$ is a function of position, or is a function of $q$? In the first
case, it is sometimes possible to define a change of variable $x(y)$
such that:
\begin{equation}
 {\rm d}y = \frac{{\rm d}x}{D(x)}
\end{equation}
In this case:
\begin{equation}
\frac{1}{D(x)} \frac{\partial}{\partial x} = \frac{\partial}{\partial
y}
\end{equation}
and then we just get:
\begin{equation}
\frac{\partial q}{\partial t} = \frac{1}{D(y)}\frac{\partial^2
q}{\partial x^2}
\end{equation}
Note that in this (and most other nontrivial cases) the stability
criterion is not the same across the whole grid. So:
\begin{equation}
\Delta t \le {\rm min} \left[\frac{(\Delta y)^2}{2 D(y)^{-1}}\right]
\end{equation}
where there is {\it not} a typo in the denominator! 

This isn't always possible but in this and in other situations one can
use a finite differencing with $D$ evaluated in the right places.
The explicit form of the finite difference becomes, for example:
\begin{equation}
 \frac{q_j^{n+1} - q_j^n}{\Delta t}
  = \frac{D_{j+1/2} \left(q_{j+1}^n - q_{j}^n\right) -
  D_{j-1/2} \left(q_j^n - q_{j-1}^n\right)}{(\Delta x)^2}
\end{equation}
Again the stability criterion will be dependent on position and we
have to pick the smallest time step (i.e. corresponding to highest
$D$ in our domain). We can do the same thing for fully implicit or
Crank-Nicolson methods. 

Nonlinear cases (where $D$ is a function of $q$) are more
difficult in general. Explicit methods can just evaluate $D$ at the
current time step:
\begin{equation}
 D_{j+1/2} = \frac{1}{2}\left[D\left(q_{j+1}^n\right) +
 D\left(q_j^n\right)\right].
\end{equation}
But then if we plug this into an implicit scheme:
\begin{equation}
 \frac{q_j^{n+1} - q_j^n}{\Delta t}
  = \frac{D_{j+1/2}\left(\vec{q}^{n+1}\right) \left(q_{j+1}^{n+1} - q_{j}^{n+1}\right) -
  D_{j-1/2}\left(\vec{q}^{n+1}\right) \left(q_j^{n+1} - q_{j-1}^{n+1}\right)}{(\Delta x)^2}
\end{equation}
this isn't a simple tridiagonal linear system any more.  The only way
is if we can now change variables in the dependent variables to
$z(q)$:
\begin{equation}
{\rm d}z = D(q) \,{\rm d}q, 
\end{equation}
which is nice because the RHS becomes:
\begin{equation}
\frac{\partial^2 z}{\partial x^2}
\end{equation}
which has a straightforward finite difference.  Great, but we are
trying to evolve $q$. To do this we linearize:
\begin{equation}
 z_j^{n+1} = z_j^n + \left(q_j^{n+1} - q_j^{n}\right) D\left(q_j^n\right)
\end{equation}
Then plugging this into the spatial finite difference, we get an
linear implicit scheme, where the nonlinear parts are basically
calculated from step $n$.

\section{Schr\"odinger Equation}

In ordinary differential equations, there was an argument that
sometimes one would trade decreased accuracy in some respects, for
increased accuracy in other respects, like for energy conservation or
phase space conservation. There are similar trade-offs in partial
differential equations.

The Schr\"odinger equation is one example, where unitarity is
important---if you are following an electron, the total probability of
that electron existing {\it somewhere} really ought to stay near
unity.  It turns out that Crank-Nicolson, in addition to its nice
properties above, {\it also} guarantees unitarity.

We'll strip out the mass and Planck's constant and just write:
\begin{equation}
 i \frac{\partial\phi}{\partial t} = - \frac{\partial^2 \phi}{\partial x^2}
+ V(x)\phi.
\end{equation}

The first-order explicit scheme for this method is unconditionally
unstable. A first-order implicit scheme yields:
\begin{equation}
i \left[\frac{\phi_j^{n+1} - \phi_j^n}{\Delta t}\right]
= - \left[\frac{\phi_{j+1}^{n+1} - 2 \phi_{j}^{n+1}
+ \phi_{j-1}^{n+1}}{\left(\Delta x\right)^2} \right] +
V_j \phi_j^{n+1}
\end{equation}
This yields:
\begin{equation}
\left[
\begin{matrix}
\beta_0 & \alpha & 0 & 0 & 0 & \ldots\cr
\alpha & \beta_1 & \alpha & 0 & 0 &\ldots\cr
0 & \alpha & \beta_2 & \alpha & 0 & \ldots\cr
\ldots & \ldots & \ldots & \ldots & \ldots & \ldots \cr
\ldots & 0 & 0 & \alpha & \beta_{N-2} & \alpha \cr
\ldots & 0 & 0 & 0 & \alpha & \beta_{N-1} \cr
\end{matrix}
\right] \cdot
\left[
\begin{matrix}
\phi_0^{n+1} \cr
\phi_1^{n+1} \cr
\phi_2^{n+1} \cr
\ldots \cr
\phi_{N-2}^{n+1} \cr
\phi_{N-1}^{n+1} \cr
\end{matrix}\right]
 =
\left[
\begin{matrix}
\phi_0^{n} \cr
\phi_1^{n} \cr
\phi_2^{n} \cr
\ldots \cr
\phi_{N-2}^{n} \cr
\phi_{N-1}^{n} \cr
\end{matrix}\right]
\end{equation}
where:
\begin{eqnarray}
\alpha &=& - \frac{\Delta t}{(\Delta x)^2} i \cr
\beta_j &=& \left[\frac{2 \Delta t}{(\Delta x)^2} + (\Delta t) V_j\right] i + 1
\end{eqnarray}
Then von Neumann yields:
\begin{equation}
i \left[\frac{\xi - 1}{\Delta t}\right] =
\xi \left[ \frac{2 \cos k \Delta x - 2}{(\Delta x)^2} - V_j \right] 
\end{equation}
and rearranging and using the same trig tricks as above:
\begin{equation}
\xi = \frac{1}{1 + i \left[\frac{4 \Delta t}{(\Delta
x)^2} \sin^2\left(\frac{k \Delta x}{2}\right)+ V_j \Delta t\right]}
\end{equation}
and this always is less than one because the modulus of the
denominator is always greater than one.

{\it But} this method is not unitary in the sense of maintaining
\begin{equation}
 \int_{-\infty}^{\infty} {\rm d}x \left|\phi(x)\right|^2 = 1
\end{equation}
This sort of issue was a
nuisance for ODEs, which motivated symplectic methods, but in the case
of Schr\"odinger it is extremely severe and makes the implicit methods
useless, as I show in the notebooks.

A unitary form of the same first order approximation can be obtained
as follows. Another way to think about the solutions to this equation
across $\Delta t$ is:
\begin{equation}
\vec{\phi}^{n+1} = \exp(- i H (\Delta t)) \vec{\phi}^n
\end{equation}
and the first-order-in-time explicit approximation of this is:
\begin{equation}
\vec{\phi}^{n+1} \approx (1 - i H (\Delta t)) \vec{\phi}^n
\end{equation}
and the first-order-in-time implicit approximation of this is:
\begin{equation}
\vec{\phi}^{n+1} \approx \left(1 + i H (\Delta t)\right)^{-1} \vec{\phi}^n
\end{equation}
where we evaluate $H$ discretely. The norm of $\phi$ will evolve as:
\begin{equation}
\left[\left(1 + i H (\Delta t)\right)^{-1}\right]^\dag
\left(1 + i H (\Delta t)\right)^{-1}  =
\left(1 - i H (\Delta t)\right)^{-1}
\left(1 + i H (\Delta t)\right)^{-1}
\end{equation}
which for some eigenfunction with eigenvalue $E$ will go as $1-(\Delta
t)^2 E^2$, and thus will cause a decrease as we saw. 

But at the same order of approximation we can write:
\begin{equation}
\vec{\phi}^{n+1} = \exp(i H (\Delta t)) \vec{\phi}^n \approx
(1 + i H (\Delta t) / 2)^{-1}
(1 - i H (\Delta t) / 2)
\vec{\phi}^n
\end{equation}
which can be thought of as moving forward half a step explicitly and
the next half step implicity. As noted above, this is just
Crank-Nicolson. This is unconditionally stable and the norm is
constant, since its adjoint operator is its inverse.
