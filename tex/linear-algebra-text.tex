\title{Linear Algebra}

These notes draw heavily on {\it Numerical Recipes}, a valuable
resource for entry-level understanding of numerical methods relevant
to physics.

\section{Linear Algebra in Computation}

A huge fraction of numerical computation involves linear algebra at
some level or another. This includes simple application of matrix
multiplication, but also applications of matrix inversion,
decompositions, and other important operations.  Vectors and
higher-dimensional objects in linear algebra are held as arrays in
NumPy (and other languages).

You are aware of the usual matrix notation:
\begin{eqnarray}
x_i &\rightarrow& \vec{x} \cr
Q_{ij} &\rightarrow& \mat{Q}
\end{eqnarray}
so for example:
\begin{equation}
y_i = \sum_j Q_{ij} x_j
\end{equation}
may also be written:
\begin{equation}
\vec{y} = \mat{Q}\cdot \vec{x}
\end{equation}

If you have an array in Python or another language, you can perform
these operations explicitly, but it is better to use the explicit
matrix operations in NumPy and associated packages. It will be far
faster. If you delve deep into high-order objects (e.g. $T_{ijkl}$)
you can definitely run into cases where the standard routines won't do
the operation you want, but most cases will work fine.

I show some examples in the workbook.

\section{Linear systems of equations}

Many physical and statistical problems boil down to solving a linear
system of the form:
\begin{equation}
\mat{A} \cdot \vec{x} = \vec{b}
\end{equation}
where $\mat{A}$ and $\vec{b}$ are known, and we want to know
$\vec{x}$.

\noindent {\bf If $\mat{A}$ is an $M\times N$ matrix (and therefore
  $\vec{x}$ is an $N$-vector) then under what conditions is there a
  unique, exact solution $\vec{x}$ to this equation?}

\begin{answer}
If $\mat{A}$ is invertible. This generally means that $\mat{A}$ has to
be $N\times N$ (so $M=N$), and that the {\it rank} of this square
matrix is $N$. Rank $N$ implies the rows (and columns) of $\mat{A}$
are all linearly independent.

Another way of looking at this is asking what is the dimension of the
space spanned by the rows of $\mat{A}$. If any row of $\mat{A}$ is a
linear combination of other rows of $\mat{A}$, then it doesn't
introduce a new linearly independent dimension.

These sorts of issues in matrices are known as {\it degeneracies}. In
these cases, there can be multiple choices of $\vec{x}$ that satisfy
the linear system. These different choices are said to be degenerate
with each other.

A matrix with these sorts of degeneracies is called {\it singular}.
\end{answer}

\noindent {\bf How can we check if matrices are singular?}

\begin{answer}
  One answer is to check the determinant:
\begin{equation}
\label{eq:det}
\det(A) = \sum_{i_0, i_1, \ldots, i_{N-1} = 1}^{N} \epsilon_{i_0, i_1,
  \ldots, i_{N-1}} a_{0,i_0} a_{1,i_1} \ldots a_{{N-1}, i_{N-1}}
\end{equation}
where we use the Levi-Civita symbol. This explicit calculation is not
{\it usually} how one calculates the determinant.

If the determinant is zero, the matrix is singular. We will see later
how this connects to the eigenvalues and eigenvectors of certain types
of matrices.
\end{answer}

You have probably done solved linear systems by hand using a
Gauss-Jordan type of technique.  This is not what is done
numerically. Instead, the simplest matrix inversion technique of
practical use is {\it LU decomposition}.

LU decomposition yields a lower triangular matrix $\mat{L}$ and and
upper triangular matrix $\mat{U}$, which satisfy:
\begin{equation}
\mat{L} \cdot \mat{U} = \mat{A}
\end{equation}

How do you do this? Look at a $3\times 3$ case:
\begin{equation}
\mat{A} = 
\left(
\begin{matrix}
A_{00} & A_{01} & A_{02} \cr
A_{10} & A_{11} & A_{12} \cr
A_{20} & A_{21} & A_{22} \cr
\end{matrix}
\right)
\end{equation}
We can multiply this by a matrix
\begin{equation}
\mat{L_0} = 
\frac{1}{A_{00}}
\left(
\begin{matrix}
1 & 0 & 0 \cr
- A_{10} & A_{00} & 0 \cr
- A_{20} & 0 & A_{00} \cr
\end{matrix}
\right)
\end{equation}
and find:
\begin{equation}
\mat{L_0}\cdot\mat{A} = 
\left(
\begin{matrix}
1 & B_{01} & B_{02} \cr
0 & B_{11} & B_{12} \cr
0 & B_{21} & B_{22} \cr
\end{matrix}
\right)
\end{equation}
Then you can then multiply by:
\begin{equation}
\mat{L_1} = 
\frac{1}{B_{11}}
\left(
\begin{matrix}
B_{11} & 0 & 0 \cr
0 & 1 & 0 \cr
0 & - B_{21} & B_{11} \cr
\end{matrix}
\right)
\end{equation}
which leads to:
\begin{equation}
\mat{L_1}\cdot\mat{L_0}\cdot\mat{A} = 
\left(
\begin{matrix}
1 & C_{01} & C_{02} \cr
0 & 1 & C_{12} \cr
0 & 0 & C_{22} \cr
\end{matrix}
\right)
\end{equation}
and then you just multiply by:
\begin{equation}
\mat{L_2} = 
\frac{1}{C_{22}}
\left(
\begin{matrix}
C_{22} & 0 & 0 \cr
0 & C_{22} & 0 \cr
0 & 0 & 1 \cr
\end{matrix}
\right)
\end{equation}
and then:
\begin{equation}
\mat{L_2}\cdot\mat{L_1}\cdot\mat{L_0}\cdot\mat{A} = 
\left(
\begin{matrix}
1 & D_{01} & D_{02} \cr
0 & 1 & D_{12} \cr
0 & 0 & 1 \cr
\end{matrix}
\right)
\end{equation}
We can then write:
\begin{equation}
\mat{A} = 
(\mat{L_0}^{-1}\cdot
\mat{L_1}^{-1}\cdot
\mat{L_2}^{-1})\cdot
(\mat{L_0}\cdot
\mat{L_1}\cdot
\mat{L_2}\cdot\mat{A}) = \mat{L}\cdot\mat{U}
\end{equation}
It is straightforward to extend this to any size matrix, so you can
create $\mat{L_i}$ and just multiply out to get $\mat{U}$. Each
$\mat{L_i}^{-1}$ can be found through backsubstitution (see below),
but actually working through this by hand and multiplying out you will find: 
\begin{equation}
\mat{L} = 
\left(
\begin{matrix}
A_{00} & 0 & 0 \cr
A_{10} & B_{11} & 0 \cr
A_{20} & B_{21} & C_{22} \cr
\end{matrix}
\right)
\end{equation}
So there is a good way to do the decomposition.

This then means we can solve:
\begin{equation}
\label{eq:lyb}
\mat{L} \cdot \vec{y} = \mat{b}
\end{equation}
and then:
\begin{equation}
\label{eq:uxy}
\mat{U} \cdot \vec{x} = \mat{y}
\end{equation}
and the resulting $\vec{x}$ satisfies:
\begin{equation}
\mat{A} \cdot \vec{x} = \mat{L} \cdot \mat{U} \cdot \vec{x}  = \mat{L}
\cdot \vec{y} = \vec{b}
\end{equation}

Hooray, but how does that help us. Well, with a triangular matrix, the
equations \ref{eq:lyb} and \ref{eq:uxy} can easily be solved by
backsubstitution.  That is, you can start at:
\begin{equation}
y_0 = \frac{b_0}{L_{00}}
\end{equation}
and then find:
\begin{equation}
y_i = \frac{b_i - \sum_{j=0}^{i-1} L_{ij} y_j}{L_{11}}
\end{equation}

\noindent {\bf How many operations does the solution take once you
  have \mat{L} and \mat{U}?}

\begin{answer}
It takes $\mathcal{O}(N^2)$. For each row, you have to do of order $N$
operations, and there are $N$ rows. 
\end{answer}

If you work through the operations above, the LU decomposition itself
takes $\mathcal{O}(N^3)$; it requires a series of $N$ matrix
multiplications basically. Now, in most implementations, in fact what
is done is that the rows of $\mat{A}$ are shuffled in the procedure so
as to keep the numerics stable.

Also, the decomposition is not unique. There are $N^2 + N$ values to
set in $\mat{L}$ and $\mat{U}$ but only $N^2$ values in $\mat{A}$. The
typical choice is to let the diagonal of $\mat{L}$ be just 1s, as done
above. 

\noindent {\bf This gives me a way of solving the equations. How do I
  determine the inverse of $\mat{A}$?}

\begin{answer}
By calling it $N$ times for each column $j$, each time with $b_i =
\delta_{ij}$. 
\end{answer}

\noindent {\bf If I have an LU decomposition of $\mat{A}$, how can I
  calculate the determinant of $\mat{A}$?}

\begin{answer}
If we have a lower or upper triangular matrix, the determinant is just
the product of all the diagonals. All other terms in the determinant
vanish, because if you involve one non-diagonal element, you end up
involving an element on the other triangle of the matrix. This is
clear in equation \ref{eq:det}, where the contribution of the diagonal
terms yields a Levi-Civita value of 1, but introducing a term with
non-diagonal factors means you need to swap two of the
indices. E.g. if I swap 2 and 3, then I get factors $a_{32}$ and
$a_{23}$. For both $\mat{L}$ and $\mat{U}$, one or the other of these
are zero.

Then we have:
\begin{eqnarray}
\det \mat{L} &=& 1 \cr
\det \mat{U} &=& \prod_{i=0}^{N-1} U_{ii}
\end{eqnarray}

And finally we use the general rule for determinants of matrix
products:
\begin{equation}
\det \mat{A} =  \det \mat{L}  \cdot \mat{U} = 
\det \mat{L} \det \mat{U}
\end{equation}
\end{answer}

I show in the notebook an example of using the {\tt linalg} routine
{\tt solve}, which utilizes this technique to solve a set of linear
equations. 

\section{Singular Value Decomposition}

The above methods are great if you have an invertible $N\times N$
matrix. But what if the solution to $\mat{A}\cdot \vec{x} = \vec{b}$
is not unique, or doesn't exist? Or, what if $\mat{A}$ is sufficiently
close to singular that round-off error starts makes the solution for
the inverse numerically unstable. What do we do? The answer is {\it
  singular value decomposition} (SVD), which provides a stable way to
deal with all of these cases.

SVD  relies on the fact that any $M\times N$ matrix can be written
as:
\begin{equation}
\mat{A} = \mat{U}\cdot\mat{W}\cdot\mat{V}^T
\end{equation}
where $\mat{U}$ is $M\times N$ with columns that are all orthonormal,
$\mat{W}$ is an $N\times N$ diagonal matrix, and $\mat{V}^T$ is the
transpose of an $N \times N$ orthonormal matrix.

The components of the diagonal matrix $\mat{W}$ are the {\it singular
  values}.  By convention in these decompositions, they are arranged
in descending order. If $M<N$ (or if for any other reason the row rank
is less than $N$), then there are fewer independent equations than
unknowns, and there will be $w_j$ values that take the value zero.

The condition of orthonormality on $\mat{V}$ means:
\begin{equation}
\mat{V}^T \cdot \mat{V} = 1
\end{equation}

And on $\mat{U}$ this means if $M\ge N$:
\begin{equation}
\mat{U}^T \cdot \mat{U} = 1
\end{equation}
but if $M<N$ then some of the $w_j$ are zero, so:
\begin{equation}
\sum_{i=0}^{M-1} U_{ik} U_{in} = \delta_{kn}
\end{equation}
for $k\le M-1$ and $n\le M-1$, but for higher $k$ and $n$ the sum
yields zero.

Performing this decomposition can be done very stably, using routines
within {\tt linalg}. Exactly how it is done is outside our scope here. 

What is the point of doing this? Consider $\mat{A}\cdot\vec{x} =
\vec{b}$. $\vec{x}$ is an $N$-dimensional vector, and $\vec{b}$ is
$M$-dimensional. $\mat{A}$ maps between these two spaces, but it may
not be able to map to the full $M$-dimensional space.  It may have the
property that it can only map to a subspace.

A trivial example is:
\begin{equation}
  \mat{A} = \left(\begin{array}{cc}
    1 & -1\cr
    -1 & 1
    \end{array}\right)
\end{equation}
Any vector $\vec{x}$ I multiple by ends up on the line $b_0 =
-b_1$. $\mat{A}$ maps from a two-dimensional space into a
one-dimensional one.

\noindent {\bf Is this ${\bf A}$ invertible?}

\begin{answer}
  Nope. Clearly any given $\vec{b}$ must map onto mulitple $\vec{x}$
  values, so it cannot be (and, accordingly, clearly has a determinant
  of zero).
\end{answer}

The {\it range} of $\mat{A}$ is the subspace $\mat{A}$ can map
into. The {\it rank} of $\mat{A}$ is the dimension of the range.  If
the rank is less than the number of dimensions ($N$) of $\vec{x}$,
then what happens to all the other dimensions? It indicates that there
is a {\it null space} for which $ \mat{A}\cdot \vec{x} = 0$. The
dimension of the null space is $N$ minus the rank.

What SVD does is basically an analysis of $\mat{A}$ which tells you
the range and null space, and the rank of the matrix. Specifically,
the columns of $\mat{U}$ corresponding to non-zero $w_j$ form an
orthonormal basis of the range of $\mat{A}$ --- clearly the ${\bf
W}\cdot\mat{V}^T$ that multiplies into $\mat{U}$ must yield something
that lives in this subspace. The columns of $\mat{V}$ corresponding to
zero $w_j$ span the null space.  The rank is the number of non-zero
$w_j$.

This decomposition gives yields a useful quantity for quantifying the
singularity of a matrix, called the {\it condition number}. The
condition number of a matrix is the ratio between the maximum and
minimum value of $w_j$. For a singular matrix this is
infinity. However, whenever this number is close to the inverse of the
machine precision (e.g. $10^{-15}$ in double precision) then the
matrix is close enough to singular that numerics on it will become
unstable. 

\section{Inversion of square matrices in SVD}

Let us return to the question of $N\times N$ matrices that we perform
SVD on.

\noindent {\bf What is the inverse of $\mat{A}$ in terms of the
decomposition?}

\begin{answer}
\begin{equation}
\mat{A}^{-1}
= \left( \mat{U} \cdot \mat{W} \cdot \mat{V}^T\right)^{-1} 
= (\mat{V}^{T})^{-1} \cdot \mat{W}^{-1} \cdot (\mat{U})^{-1} 
= \mat{V} \cdot \mat{W}^{-1} \cdot \mat{U}^{T} 
\end{equation}
\end{answer}

Clearly, then, if \mat{A} is not singular or close to singular, once
the SVD is done it is trivial to invert the matrix \mat{A}, and obtain
the solution to the problem.

Now, what if the matrix \mat{A} is singular? If so, then there are two
possibilities. Either $\vec{b}$ is in the range of \mat{A}, or it is
not.

If it is in the range, then there are an infinite set of solutions; if
you find one solution, you can always take that solution $\vec{x}$
plus any solution in the null space $\vec{x}'$, and it is a new
solution.  You can find the solution with the smallest norm in the
following way. Take all the zero $w_j$, and replace $1/w_j$ with 0
instead of infinity. What??? Yes.

You can prove this as follows. Let's take the $\vec{x}$ that we get
from the pseudoinverse, i.e. where $\mat{W}^{-1}$ is the inverse
of \mat{W} with the adjustment made for zero $w_j$. Then let us
perturb it with $\vec{x}'$ in the null space. Then we want to ask,
what choice of $\vec{x}'$ will minimize the norm of $\vec{x}$? So
let's calculate the norm:
\begin{eqnarray}
|\vec{x} + \vec{x}'|
&=& \left| \mat{V}\cdot\mat{W}^{-1} \cdot \mat{U}^T\cdot \vec{b}
+ \vec{x}'\right| \cr
&=& \left| \mat{V}\cdot \left(\mat{W}^{-1} \cdot \mat{U}^T\cdot \vec{b}
+ \mat{V}^T \cdot \vec{x}' \right)\right| \cr
&=& \left| \mat{W}^{-1} \cdot \mat{U}^T\cdot \vec{b}
+ \mat{V}^T\cdot \vec{x}'\right| 
\end{eqnarray}
where the last equality follows from the orthonormality of \mat{V}.
The left term has components in coordinate directions only where
$w_j\ne 0$. The right term is in the null space, rotated by
$\mat{V}^T$, so has only components in directions where
$w_j=0$. Therefore 
\begin{eqnarray}
|\vec{x} + \vec{x}'|^2
&=& \left| \mat{W}^{-1} \cdot \mat{U}^T\cdot \vec{b}\right|^2
+ \left| \mat{V}^T\cdot \vec{x}'\right|^2
\end{eqnarray}
i.e. minimizing this norm requires $\vec{x}'=0$. The $\vec{x}$ from
the pseudoinverse minimizes the norm.

If $\vec{b}$ is not in the range, then the same procedure, it turns
out, yields an $\vec{x}$ then minimizes:
\begin{equation}
r = \left|\mat{A}\cdot\vec{x} - \vec{b}\right|
\end{equation}
To see this, we again perturb $\vec{x}$ by $\vec{x}'$ (this time any
$\vec{x}'$). $\mat{A}\cdot\vec{x}' = \vec{b}'$ will live in the range
of $\mat{A}$.

Now let us consider the residual in this case:
\begin{eqnarray}
r^2 &=& \left|\mat{A}\cdot\left(\vec{x}
+ \vec{x}'\right)- \vec{b}\right|^2 \cr
 &=& \left|\left(\mat{A}\cdot\vec{x} - \vec{b}\right)
+ \vec{b}'\right|^2 \cr
 &=& \left|\left(\mat{U}\cdot\mat{W}\cdot\mat{W}^{-1}\cdot\mat{U}^T\right)\cdot\vec{b}
+ \vec{b}'\right|^2 \cr
 &=& \left|\mat{U}\cdot\left[\left(\mat{W}\cdot\mat{W}^{-1}
 - \mat{1}\right)\cdot\mat{U}^T\cdot\vec{b}
+ \mat{U}^T\vec{b}'\right]\right|^2 \cr
 &=& \left|\left(\mat{W}\cdot\mat{W}^{-1}
 - \mat{1}\right)\cdot\mat{U}^T\cdot\vec{b}
+ \mat{U}^T\vec{b}'\right|^2
\end{eqnarray}
I don't think this argument from NR is correct, since this residual
will be zero for a non-single matrix.


Therefore, the inverse with the conversion of $w_j = 0\rightarrow
1/w_j = 0$ yields what we usually want in all cases. It is known as
the {\it Moore-Penrose inverse} or {\it pseudoinverse} of a matrix. 

The SVD solution of linear systems is in almost all cases the correct
approach. Note that when $w_j$ values become less than the machine
precision, they often should also be treated as zeros to avoid
numerical instability.

\section{Sparse matrices}

There are many applications where you have in principle very large
matrices, but they are in fact very sparse. A common case is a
tridiagonal or band-diagonal system. However, it is also possible that
there are simply sparse off-diagonal terms. 

For these systems, you can save a lot of time and memory by not
operating on the full matrix, but only the non-sparse values.

SciPy has a whole set of routines for dealing with such matrices,
including inverses, SVD, etc. 

