\documentclass[11pt, preprint]{aastex}

\usepackage{environ}
\usepackage{xcolor}
\usepackage{hyperref}
\usepackage{rotating}  
\usepackage{amsmath}

\newcommand{\todo}[1]{{\bf #1}}
\newcommand{\dd}[1]{\ensuremath{{\rm d}#1}}
\newcommand{\mat}[1]{\ensuremath{{\bf #1}}}
\DeclareMathOperator{\sech}{sech}

\newif\ifanswers

\NewEnviron{answer}{\ifanswers\color{blue}\expandafter\BODY\fi}

\newenvironment{ditemize}
{ \begin{list}{}{%
\setlength{\topsep}{0pt}% 
\setlength{\partopsep}{3pt}% 
\setlength{\itemsep}{1pt}\setlength{\parsep}{1pt}% 
\setlength{\itemindent}{0pt}\setlength{\listparindent}{12pt}%
\setlength{\leftmargin}{24pt}\setlength{\rightmargin}{0in}%
\setlength{\labelsep}{6pt}\setlength{\labelwidth}{6pt}%
\renewcommand{\makelabel}{\makebox[\labelwidth][l]{$\bullet$\hspace{\fill}}}}}
{\end{list}}


\begin{document}

\title{\bf Computational Physics / PHYS-GA 2000 / Problem Set \#8
\\ Due November 5, 2024 }

You {\it must} label all axes of all plots, including giving the {\it
  units}!!

\begin{enumerate} 
\item We will do a simple likelihood maximization problem. Let's say
  you do a survey of the population, asking people a simple yes or no
  question, namely ``Do you recognize the phrase `Be Kind, Rewind',
  and know what it means?'' You have a hypothesis that whether people
  answer yes should depend on age. The standard way people analyze the
  results to look for a correlation in situations like this is
  something called {\it logistic regression.} You model the
  probability as the logistic function:
  \begin{equation}
    p(x) = \frac{1}{1+ \exp\left[-(\beta_0 + \beta_1 x)\right]}
  \end{equation}
  where in this case $x$ represents the age.  You'll see that
  $p(x)=0.5$ at $-\beta_0/\beta_1$ and the slope there is $1/\beta_1$.
  Then the likelihood for each resulting value is $p({\rm age})$ if
  the answer is ``yes'' and $1 - p({\rm age})$ if the answer is
  ``no.''  Use
  \href{https://github.com/blanton144/computational-grad/blob/main/data/survey.csv}{this
    data}, which gives the age of the respondent in years and their
  answer to the question, where ``1'' means yes and ``0'' means
  no. Note this is {\it not} from a real survey!!  Find the maximum
  likelihood values and formal errors and covariance matrix of
  $\beta_0$ and $\beta_1$. Plot the logistic model and the answers on
  the same plot. Some points to keep in mind:
  \begin{itemize}
    \item Using {\tt jax}'s {\tt autodiff} really helps for both the
      gradient and the Hessian calculation. 
    \item Remember you want to minimize the negative log likelihood.
    \item Be careful about machine precision---you need to calculate
      the log likelihood in such a way that you are never taking the
      log of zero, which if you are not careful you can easily do.
  \end{itemize}
  Do the answers make any sense?

  Note that although this logistic function regression is {\it
    standard} for problems where you are looking for a relationship
  between the probabilities from a bunch of 0s and 1s, just like
  fitting a line to data points is standard, that doesn't mean the
  logistic function is always the most sensible or correct model.
  (Though by construction it is in {\it this} case).
\item Exercise 7.3 of Newman (n.b. you may use {\tt numpy} or {\tt
  scipy}'s FFT routines).
\item Exercise 7.4 of Newman (n.b. you may use {\tt numpy} or {\tt
  scipy}'s FFT routines).
\end{enumerate} 


\end{document}
