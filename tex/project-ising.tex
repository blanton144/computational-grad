\documentclass[11pt, preprint]{aastex}

\usepackage{environ}
\usepackage{xcolor}
\usepackage{hyperref}
\usepackage{rotating}  
\usepackage{amsmath}

\newcommand{\todo}[1]{{\bf #1}}
\newcommand{\dd}[1]{\ensuremath{{\rm d}#1}}
\newcommand{\mat}[1]{\ensuremath{{\bf #1}}}
\DeclareMathOperator{\sech}{sech}

\newif\ifanswers

\NewEnviron{answer}{\ifanswers\color{blue}\expandafter\BODY\fi}

\newenvironment{ditemize}
{ \begin{list}{}{%
\setlength{\topsep}{0pt}% 
\setlength{\partopsep}{3pt}% 
\setlength{\itemsep}{1pt}\setlength{\parsep}{1pt}% 
\setlength{\itemindent}{0pt}\setlength{\listparindent}{12pt}%
\setlength{\leftmargin}{24pt}\setlength{\rightmargin}{0in}%
\setlength{\labelsep}{6pt}\setlength{\labelwidth}{6pt}%
\renewcommand{\makelabel}{\makebox[\labelwidth][l]{$\bullet$\hspace{\fill}}}}}
{\end{list}}


\begin{document}

\title{\bf Computational Physics Project / 2D Ising Models}

This project focuses on the simulation through Monte Carlo methods of
2D Ising models for magnetization. This project is similar to Exercise
10.9 in Newman.  The presentation here comes mostly from the
computational physics textbook of Landau, P{\'a}ez, \& Bordeianu (not
{\it the} Landau).

These model a magnetic material as a 2D lattice of atoms whose
magnetic moments can either be up or down. These are thermodynamic
systems, so depending on the temperature, there is more or less random
flipping of these moments.

Each neighboring pair of atoms (labeled $i$ and $j$, where $j$ is a
neighbor in $x$ or in $y$, but not along the diagonal) contributes to
the total energy of the system an amount:
\begin{equation}
  E = - J s_{i} s_{j}
\end{equation}
Aligned moments are lower energy, therefore are the preferred state.

Since the system is thermodynamic, the spins will sometimes flip
randomly. Flips that happen to be favorable happen at a constant rate,
but those that are unfavorable do so at a reduced rate; reduced
specifically by the Boltzman factor:
\begin{equation}
  R = \exp\left(-\Delta E /k T\right)
\end{equation}

We are interested in what happens to the magnetization of a system as
a function of temperature.  The magnetization can be defined as :
\begin{equation}
M = \sum_i s_i
\end{equation}

It happens that in this case there is a simple analytic form for the
magnetization as a function of temperature:
\begin{equation}
  M(T) = \left\{
  \begin{array}{ll}
  0 & T > T_c \cr
  \frac{(1+z^2)^{1/4} (1-6 z^2 +z^4)^{1/8}}{\sqrt{1-z^2}} & T < T_c
    \end{array}\right. 
\end{equation}
For $kT_c \approx 2.269185 J$ and $z=\exp(-2J/kT)$.

\section{Prep work}

\begin{itemize}
\item Explain in words why you should find $M\sim 0$ for sufficiently
  high $T$, and why you should find an extremal $M$ (all spins
  aligned) for sufficiently low $T$.
\item Read and briefly explain the Metropolis algorithm described in
  Newman section 10.3.2. We have not discussed this in class!
\item Rescale the problem so that you only need one physical constant
  (which will be a combination of $J$, $k$, and $T$). Below, perform
  the numerical analysis in those rescaled units.
\end{itemize}

\section{Writing the code}

\begin{itemize}
\item Write a piece of code to calculate the total energy of the
  system (similar to part (a) of Exercise 10.9). 
\item Write into code the Metropolis algorithm found in section 10.3.2
  of Newman. Note that you do {\it not} need to recalculate the full
  energy of the system at each trial, since most of it is unchanged
  (i.e. you only need the terms involving the lattice atom you
  picked). Use periodic boundary conditions.
\end{itemize}

\section{Testing and running the code}

\begin{itemize}
\item Now initialize your system to start with a fully magnetized (all
  spins aligned, with $s_i = 1$) system, at $T=0$. Raise the
  temperature in small steps, and after each step run the Metropolis
  algorithm for long enough that the energy and magnetization seem to
  have converged. Use the final conditions of one temperature step as
  initial conditions for the next temperature step.
\item Plot the magnetization and energy as a function of temperature
  and compare the magnetization to the analytic prediction.
\item What is the physical meaning of the slope of the energy versus
  temperature?  Can you describe in words what this slope means in
  terms of how much work it is to raise the temperature?
\item If you add a term to $E$ that accounts for an external magnetic
  field:
  \begin{equation}
    -H \sum_i s_i
  \end{equation}
  how does this change the behavior of the system when you start with
  $T=0$ and raise the temperature?  Try both $H$ positive and
  negative.
\item Try a few runs where you start with random spin orientations and
  $T\gg T_c$, and cool the system down. What do you notice about the
  final result? What if you include a non-zero $H$?
\item Test the fluctuation-dissipation theorem, which says that the
  rms variation of the energies will be proportional to $E/C_V$. You
  can test this both by varying the size of the system and by
  considering the fluctuations at different points in the curve.
\end{itemize}

\section{Bonus}

\begin{itemize}
% Based on Lane & Shi project 2023
\item Test the effect of a sinusoidally varying field, of various
  periods. Comment on the results.
\end{itemize}


\end{document}
