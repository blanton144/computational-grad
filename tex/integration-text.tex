\title{Integration}

\section{Basic idea of integration}

Integration of functions is an important calculation in computational
physics, both as a fundamental task and as a component of larger
problems. Many integrals do not have closed forms and require
numerical computation.

\noindent {\bf What is the definition of an integral?}

\begin{answer}
An integral is defined by the limit:
\begin{equation}
\int_a^b {\rm d}x f(x) = \lim_{{\rm d}x \rightarrow 0} \left[{\rm d}x
  \sum_{i=1}^{(b-a)/{\rm d}x} f(x_i) \right]
\end{equation}
where $x_i$ are spaced between $a$ and $b$ with separations ${\rm
  d}x$.
\end{answer}

\noindent {\bf What is a simple numerical estimate of an integral?}

\begin{answer}
Just to perform this sum with some finite ${\rm d}x$:
\begin{equation}
\int_a^b {\rm d}x f(x) = {\Delta}x \left[
\frac{1}{2} \left(f(x_0) + f(x_1)\right)
\sum_{i=1}^{(b-a)/{\Delta}x - 1} f(x_i) \right]
\end{equation}
where $x_i$ are spaced between $a + \Delta x / 2$ and $b - \Delta x
 /2$ with separations ${\rm d}x$.

You can think of this as an integration of a piecewise constant
interpolation of the function.
\end{answer}

The error in this calculation can be evaluated by expanding the real
integral as a Taylor series. For one $x_i$, the integral is:
\begin{eqnarray}
I &=& \int_{x_i-\Delta x/2}^{x+\Delta x/2} {\rm d}x f(x) \cr
&=& \int_{x_i-\Delta x/2}^{x+\Delta x/2} {\rm d}x
\left\{f(x_i) +
(x - x_i) \left.\frac{\partial f}{\partial x}\right|_{x_i} +
\frac{1}{2} (x - x_i)^2 
\left.\frac{\partial^2 f}{\partial x^2}\right|_{x_i} +
\mathcal{O}\left[(x-x_i)^3\right]\right\} \cr
&=& \Delta x f(x_i) + 
\frac{1}{6} \Delta x^3
\left.\frac{\partial^2 f}{\partial x^2}\right|_{x_i} +
\mathcal{O}\left[\Delta x^4\right]
\end{eqnarray}
The first order term in the Taylor series drops because of symmetry.
For an integration over some fixed range, the number of steps
$N\propto \Delta x^{-1}$, so the error term leftover is $\Delta x^2$.

This is just a particular case of the more general form that most
integration methods take, which is that it can be approximated as some
linear combination of evaluations of the function:
\begin{equation}
  \int_a^b {\rm d}x f(x) = \sum_{i=1}^N 
  f(x_i) w_i
\end{equation}

\section{Trapezoid rule}

The simple estimate above can be thought of as approximating the
function as piecewise constant. Obviously there are better
approximations that can be made! Better algorithms for integration
generally boil down to better models of the function. In this respect,
integration is closely allied to interpolation of functions.

The trapezoid rule is the result of integrating a linear interpolation
of the function. Each term in the integral will become:
\begin{equation}
\frac{1}{2} \Delta x \left( f_i + f_{i+1} \right) 
\end{equation}
The next term is:
\begin{equation}
\frac{1}{2} \Delta x \left( f_{i+1} + f_{i+2} \right) 
\end{equation}
For equally spaced points, then $w_i = \Delta x$, except for $w_1=
w_{N} = \Delta x/2$.

\noindent {\bf For what sort of function is the trapezoid rule exactly
  correct?}

\begin{answer}
For a linear function. Of course, this property is not very useful!!
\end{answer}

But \ldots clearly the trapezoid rule and the piecewise constant are
the same, except for the treatment of the end points. What is going
on? What is going on is that the linear interpolation introduces a
linear term in the approximation, but over the integration interval,
that linear term integrates to zero because it is odd. So really a 0th
order polynomial interpolation of the function (piecewise constant)
gives the same precision in the integral as a 1st order polynomial
interpolation of the function.

This explanation might be a bit confusing because, unlike the
piecewise constant case, in the linear interpolation case the
approximating integral isn't symmetric around $x_i$, but around
$x_i+\Delta x / 2$, and we do not directly have $f(x_i + \Delta x
/2)$---just an estimate of it as the average of the two end points.
But let's look at the error in that estimate:
\begin{eqnarray}
\hat f_{i+1/2} &=& \frac{1}{2} \left[f_i + f_{i+1}\right]\cr
&=& \frac{1}{2} \left[
f_{i+1/2} - \frac{\Delta x}{2} \left.\frac{\partial f}{\partial x}\right|_{x_{i+1/2}} +
 \mathcal{O}\left(\Delta x^2\right) + \right. \cr
 & & 
\left. f_{i+1/2} + \frac{\Delta x}{2} \left.\frac{\partial f}{\partial x}\right|_{x_{i+1/2}} +
 \mathcal{O}\left(\Delta x^2\right) \right] \cr
&= & f_{i+1/2} + \mathcal{O}\left(\Delta x^2\right)
\end{eqnarray}
That is, the errors in this estimate are $\mathcal{O}(\Delta x^2)$ so
the errors in each term remain $\mathcal{O}(\Delta x^3)$ just as in
the piecewise constant case (which must be the case since both methods
are the same except for the end points).

In either case we expect the remainder error to scale with the
next-order non-zero term in a polynomial expansion, or second order in
this case. This means that in each term of the sum the next-order
missing term is of order $\sim \Delta x^2$, which should tell us the
approximation error. In the notebook examples we can explore this
scaling.

\section{Simpson's rule}

Simpson's rule represents the next level of sophistication in
interpolation. Here, the function is approximated locally around the
points $i-1$, $i$, $i+1$, as a quadratic:
\begin{equation}
f(x) = \alpha' + \beta' x + \gamma' x^2
\end{equation}
This is not a very convenient form. Let us instead use:
\begin{equation}
  f(x) = \alpha + \beta \left(\frac{x - x_i}{{\rm d}x}\right) +
  \gamma \left(\frac{x - x_i}{{\rm d}x}\right)^2 = 
  \alpha + \beta y
  + \gamma y^2
\end{equation}
with a change of variable to $y = (x-x_i)/{\rm d}x$.  For a set of
three points, $i-1$, $i$, and $i+1$, you can fit the parabola using
the fact:
\begin{eqnarray}
f_{i-1} &=& \alpha - \beta + \gamma \cr
f_{i} &=& \alpha \cr
f_{i+1} &=& \alpha + \beta + \gamma
\end{eqnarray}
This can be easily solved:
\begin{eqnarray}
\alpha &=& f_i \cr
\gamma &=& \frac{f_{i+1}+f_{i-1}}{2} - f_i \cr
\beta &=& \frac{f_{i+1}-f_{i-1}}{2}
\end{eqnarray}

\noindent {\bf What is the integral over the region defined by these
  three points?}

\begin{answer}
The integral over the region defined by these three points :
\begin{eqnarray}
  \int_{x_{i-1}}^{x_{i+1}} {\rm d}x f(x) &=& {\rm d}x \int_{-1}^{1} {\rm
    d}y \left(\alpha + \beta y + \gamma y^2\right) \cr
  &=& {\rm d}x \left[\alpha y + \frac{\beta}{2} y^2 + \frac{\gamma}{3}
    y^3\right]_{-1}^{1}  \cr
  &=& {\rm d}x \left[2 \alpha + \frac{2\gamma}{3}\right]
\end{eqnarray}
Plugging in $\alpha$ and $\gamma$:
\begin{equation}
  \int_{x_{i-1}}^{x_{i+1}} {\rm d}x f(x)
  = {\rm d}x \left(2 f_i +
  \frac{f_{i+1} + f_{i-1}}{3} - \frac{2}{3} f_i\right) 
  = {\rm d}x \left(\frac{1}{3} f_{i-1}
  + \frac{4}{3} f_i 
  + \frac{1}{3} f_{i+1}\right)
\end{equation}
\end{answer}

Simpson's rule comes from using this approximation across the length
from $a$ to $b$, by dividing the interval into an even number of
segments, and integrating each separately. This yields a full
summation:
\begin{equation}
  \int_a^b {\rm d}x f(x) = \sum_{i=1}^N = {\rm d}x \left[\frac{1}{3}
    f_1 + \frac{4}{3} f_2 + \frac{2}{3} f_3 + \frac{4}{3} f_4 + \ldots
    + \frac{2}{3} f_{N-2} + \frac{4}{3} f_{N-1} + \frac{1}{3} f_N
    \right]
\end{equation}

Because this is applied to two segments at a time, it requires an even
number of segments, which means $N$ must be odd.

\noindent {\bf The weights for the three points used in in Simpson's
  rule are set to exactly integral a quadratic function --- a second
  degree polynomial. What must $N$ be to exactly integrate an
  $M$-degree polynomial?}

\begin{answer}
$N = M + 1$. We can show this as follows. Each of the $N$ points $x_k$
  yields a linear equality:
\begin{equation}
\label{eq:weights}
 f_k = \sum_{j=0}^M \alpha_j x^j
\end{equation}
that can be used to determine the coefficients of the function, and
thus its integral.  So this yields a system of $N$ linear equations,
with $M+1$ unknowns. So to guarantee a solution, you need $N=M+1$.
\end{answer}

Note that the $\beta$ term integrated out. This is for the same reason
that the linear term integrated out in the trapezoidal rule, making it
the same as the piecewise constant rule. Consider now a cubic
approximation:
\begin{equation}
  f(x) = \alpha + \beta y + \gamma y^2 + \epsilon y^3
\end{equation}
Let's say we interpolated the function with these parameters and then
integrated. This time the linear {\it and} cubic terms would drop out.

This means that the Simpson's rule approximation error doesn't involve
any cubic terms in the interpolation function, but quartic terms. This
term scales as $\Delta x^4$, and we can use the notebook to show how
this scaling works.

There's another way to think about Simpson's rule, which will be a
useful way to think about dealing with approximation errors in ODEs
too, when we get there. Let's take the trapezoidal rule. We know it
yields an approximation $\hat I_0$ that differs from the real integral
$I$ as follows:
\begin{equation}
\hat I_0 = I + \epsilon \Delta x^2 + \mathcal{O}\left(\Delta x^4\right)
\end{equation}
We don't know $\epsilon$, but it does not depend on $\Delta x$ at
leading order. Let us take a second estimate of $I$ with half the step
size, i.e. $\Delta x /2$, which would satisfy: 
\begin{equation}
\hat I_1 = I + \frac{1}{4} \epsilon \Delta x^2 + \mathcal{O}\left(\Delta x^4\right)
\end{equation}
Having done this we can combine these results and cancel the leading error
term:
\begin{equation}
\hat I = \frac{4}{3} I_1 - \frac{1}{3} I_0 = I + \mathcal{O}\left(\Delta x^4\right)
\end{equation}

In terms of the denser sampling points, the weights are:
\begin{eqnarray}
w_1 &=& \left\{\frac{1}{2}, 0, 1, 0, \ldots, 0, 1,
0, \frac{1}{2}\right\} \times \Delta x\cr
w_2 &=& \left\{\frac{1}{2}, 1, 1, 1, \ldots, 1, 1,
1, \frac{1}{2}\right\} \times \frac{\Delta x}{2}
\end{eqnarray}
or:
\begin{eqnarray}
w_1 &=& \left\{1, 0, 2, 0, \ldots, 0, 2,
0, 1\right\} \times \frac{\Delta x}{2}\cr
w_2 &=& \left\{\frac{1}{2}, 1, 1, 1, \ldots, 1, 1,
1, \frac{1}{2}\right\} \times \frac{\Delta x}{2}
\end{eqnarray}
so when they are combined we find:
\begin{eqnarray}
w &=& \left\{\frac{1}{3}, \frac{4}{3}, \frac{2}{3}, \frac{4}{3},
\ldots, \frac{4}{3}, \frac{2}{3}, \frac{4}{3}, \frac{1}{3}\right\}  \times \frac{\Delta x}{2}
\end{eqnarray}
That is, this combination {\it is} just Simpson's rule.

Note that this also gives a way to assess the errors, because you can
estimate:
\begin{equation}
\epsilon \Delta x^2 = \frac{4}{3}  \left(\hat I_1 - \hat I_0\right)
\end{equation}
This is a simple example of checking the convergence of the
calculation. Also, in the case that you actually can call the function
(and don't just have a fixed set of values) this provides a
straightforward way of refining the step size to some desired
precision.

\section{Romberg Integration}

Simpson's method can be extended, through a version of something
called Richardson extrapolation, which in this context takes the name
of Romberg integration. In general, Richardson extrapolation means
that you perform a calculation with smaller and smaller steps (or
whatever sets your approximation error) and then use that series to
extrapolate to zero step size.  I find most explanations of this
method horribly confusing and actually somewhat vague. I'll try to do
better here.

The idea is that if you use the trapezoid method with step size
$\Delta x$, it has an error series like so:
\begin{eqnarray}
\hat I &=& I + \epsilon_0 \left(\Delta x\right)^2 + 
\epsilon_1 \left(\Delta x\right)^4 + 
\epsilon_2 \left(\Delta x\right)^6 + \ldots \cr
&=& \sum_{j=0}^{\infty} \epsilon_j \left(\Delta x\right)^{2j+1}
\end{eqnarray}
Now let's say that we have a series of $N$ trapezoid method
calculations, labeled $j=0\ldots N-1$, each with a spacing $\Delta x /
2^j$. That gives us a series:
\begin{eqnarray}
\hat I_0 &=& I + \epsilon_0 \left(\Delta x\right)^2 + 
\epsilon_1 \left(\Delta x\right)^4 + 
\epsilon_2 \left(\Delta x\right)^6 + \ldots \cr
\hat I_1 &=& I + \frac{1}{4} \epsilon_0 \left(\Delta x\right)^2 + 
\frac{1}{16} \epsilon_1 \left(\Delta x\right)^4 + 
\frac{1}{64} \epsilon_2 \left(\Delta x\right)^6 + \ldots \cr
&\ldots& \cr
\hat I_{N-1} &=& I + \frac{1}{2^{2N}} \epsilon_0 \left(\Delta x\right)^2 + 
\frac{1}{2^{4N}} \epsilon_1 \left(\Delta x\right)^4 + 
\frac{1}{2^{6N}} \epsilon_2 \left(\Delta x\right)^6 + \ldots
\end{eqnarray}
Our goal will be to determine $I$. We can't determine it
independently, or at the highest possible precision, but the series of
$N$ measurements allows us to make an approximation that retains just
$N$ unknowns and solve for all of them, including $I$. The
approximation is to cut off the series:
\begin{eqnarray}
\hat I_0 &=& \hat I + \epsilon_0 \left(\Delta x\right)^2 + 
\epsilon_1 \left(\Delta x\right)^4 + 
\epsilon_2 \left(\Delta x\right)^6
+ \ldots + \epsilon_{N-2} \left(\Delta x \right)^{2(N-1)}\cr
\hat I_1 &=& \hat I + \frac{1}{4} \epsilon_0 \left(\Delta x\right)^2 + 
\frac{1}{16} \epsilon_1 \left(\Delta x\right)^4 + 
\frac{1}{64} \epsilon_2 \left(\Delta x\right)^6 + 
+ \ldots + \frac{1}{2^{2(N-1)}} \epsilon_{N-2} \left(\Delta x \right)^{2(N-1)}\cr
&\ldots& \cr
\hat I_{N-1} &=& \hat I + \frac{1}{2^{2N}} \epsilon_0 \left(\Delta x\right)^2 + 
\frac{1}{2^{4N}} \epsilon_1 \left(\Delta x\right)^4 + 
\frac{1}{2^{6N}} \epsilon_2 \left(\Delta x\right)^6 + \ldots
+ \ldots + \frac{1}{2^{2(N-1)(N-1)}} \epsilon_{N-2} \left(\Delta x \right)^{2(N-1)}
\end{eqnarray}
where $\hat I$ represents an estimate of the integral with all error
terms up to $\Delta x^{2(N-1)}$ canceled, i.e. with error terms of
order $\Delta x^{2N-1}$: 
\begin{equation}
\hat I = I + \epsilon_{N-1} \left(\Delta x\right)^{2N-1}
+ \epsilon_{N} \left(\Delta x\right)^{2N} + \ldots
\end{equation}
The set of equations above forms a linear system with $N$ equations
and $N$ unknowns, which are $\{\hat
I, \epsilon_0, \epsilon_1, \ldots, \epsilon_{N-2}\}$. With the
substitution $y_k=(\Delta x)^2/2^{2k}$, we can rewrite the equations as: 
\begin{eqnarray}
\hat I_0 &=& \hat I +
\epsilon_0 y_0 + 
\epsilon_1 y_0^2 +
\epsilon_2 y_0^3
+ \ldots + \epsilon_{N-2} y_0^{N-1}\cr
\hat I_1 &=& \hat I + 
\epsilon_0 y_1 + 
\epsilon_1 y_1^2 +
\epsilon_2 y_1^3
+ \ldots + \epsilon_{N-2} y_1^{N-1}\cr
&\ldots& \cr
\hat I_{N-1} &=& \hat I +
\epsilon_0 y_{N-1} + 
\epsilon_1 y_{N-1}^2 +
\epsilon_2 y_{N-1}^3
+ \ldots + \epsilon_{N-2} y_{N-1}^{N-1}
\end{eqnarray}
Rewriting as a matrix equation:
\begin{equation}
\left(\begin{array}{ccccc}
1 & y_0 & y_0^2 & \ldots & y_0^{N-1} \cr
1 & y_1 & y_1^2 & \ldots & y_1^{N-1} \cr
1 & y_2 & y_2^2 & \ldots & y_2^{N-1} \cr
\ldots & \ldots & \ldots & \ldots & \ldots \cr
1 & y_{N-1} & y_{N-1}^2 & \ldots & y_{N-1}^{N-1} \cr
\end{array}\right)
\cdot
\left(\begin{array}{c}
\hat I \cr
\epsilon_0 \cr
\epsilon_1 \cr
\ldots \cr
\epsilon_{N-2}
\end{array}\right)
= 
\left(\begin{array}{c}
\hat I_0 \cr
\hat I_1 \cr
\hat I_2 \cr
\ldots \cr
\hat I_{N-1}
\end{array}\right)
\end{equation}
Now the strategy is apparent! We take a sequence of trapezoid
estimates of decreasing step size, and then fit the sequence of
integral estimates to a polynomial model in $y$. The first coefficient
ends up being the integral estimate with only error terms of order
$\Delta x^{2N-1}$ and higher.

You will usually see this described as ``the integral estimate
extrapolated to zero step size,'' which is true-ish, in the sense that
you evaluate the polynomial in $y$ at $y=\Delta x=0$. But to me that
description obscures how you have determined the coefficients of the
polynomial.

However, we do not generally want to solve this matrix equation
through a large matrix inversion---that will get expensive fast! As we
will learn later, the matrix inversion takes $\mathcal{O}(N^3)$ time
typically.  Instead, we use the same techniques we used in
interpolation, and we evaluate:
\begin{equation}
f(y=0) = \sum_{i=0}^{N-1} W(y=0, y_i) \hat I_i
\end{equation}
where:
\begin{equation}
W(y, y_i) = \prod_{j\ne i} \frac{y - y_j}{y_i - y_j}
\end{equation}
Proceeding with this calculation, we find: 
\begin{eqnarray}
W(0, y_i) &=& \prod_{j\ne i} \frac{y - y_j}{y_i - y_j} \cr
 &=& \prod_{j\ne i} \frac{- 1}{y_i / y_j - 1} \cr
 &=& \prod_{j\ne i} \frac{- 1}{2^{2j} / 2^{2i} - 1} \cr
 &=& \prod_{j\ne i} \frac{1}{1 - 2^{2(j-i)}}
\end{eqnarray}

So for $N=1$, our integral estimate is:
\begin{equation}
\hat I = W(y=0, y_0) \hat I_0 = \hat I_0
\end{equation}
And for N=2, it becomes:
\begin{eqnarray}
\hat I &=& W(y=0, y_0) \hat I_0 + W(y=0, y_1) \hat I_1 \cr
&=& \frac{1}{1 - 2^{2(1-0)}} \hat I_0 + \frac{1}{1 - 2^{2(0-1)}}\hat I_1 \cr
&=& - \frac{1}{3} \hat I_0 + \frac{1}{1 - \frac{1}{4}}\hat I_1 \cr
&=& - \frac{1}{3} \hat I_0 + \frac{4}{3}\hat I_1
\end{eqnarray}
Ta da! Simpson's rule again.

To round things out in terms of implementation, when we add the $N$th
refinement to the integration, this multiplies another factor into
each term $i$, which will be:
\begin{equation}
\frac{1}{1 - 2^{2(N-i)}}
\end{equation}
and it adds another term $\hat I_N$, which will be multiplied by:
\begin{equation}
W(0, y_N) = \prod_{j<N} \frac{1}{1 - 2^{j-N}}
\end{equation}
This leads to a very simple way to implement Romberg.

These methods are good methods, but it turns out we can be even
cleverer. But before we do so, we have a little bit of work to do.

\section{Rescaling of integrals}

It may appear trivial, but just as in differentiation, there are
rescaling of integrals that can be performed for various reasons of
convenience or otherwise. 

The simplest rescaling is linear, which just rescales the limits of
the integral:
\begin{equation}
 I = \int_a^b {\rm d}x f(x) = \frac{b-a}{b'-a'} \int_{a'}^{b'} {\rm d}x' f(x(x'))
\end{equation}
which simply follows from the tranformation:
\begin{eqnarray}
x' &=& (x-a) \left(\frac{b' - a'}{b-a}\right) + a' \cr
{\rm d}x' &=& {\rm d}x \left(\frac{b' - a'}{b-a}\right)
\end{eqnarray}
or:
\begin{eqnarray}
x &=& (x'-a') \left(\frac{b - a}{b'-a'}\right) + a' \cr
{\rm d}x &=& {\rm d}x' \left(\frac{b - a}{b'-a'}\right)
\end{eqnarray}

This is a pretty trival rescaling, but it can be useful if you can
rescale an integral to a previously calculated integral. We will use
this below in the specific case: $a'=-1$, $b'=1$:
\begin{equation}
 I = \int_a^b {\rm d}x f(x) = \frac{b-a}{2} \int_{-1}^{1} {\rm d}x' f(x(x'))
\end{equation}
This will allow us to develop some useful algorithms for the specific
range $-1$ to $1$, which can then be generalized to any finite range.

If we want to alter $[-1, 1]$ to an infinite range that is possible
too.  For example:
\begin{eqnarray}
x &=& q \frac{1+x'}{1-x'}  \cr
{\rm d}x &=& q \left(\frac{1}{1-x'} + \frac{1+x'}{(1-x')^2}\right){\rm
d}x'\cr
&=& q {\rm d}x' \frac{1 -x' + 1 + x'}{(1-x')^2}  \cr
&=& {\rm d}x' \frac{2q}{(1-x')^2}  \cr
\end{eqnarray}
which lets us rewrite an infinite range:
\begin{equation}
 I = \int_0^\infty {\rm d}x f(x) = \int_{-1}^{1} {\rm d}x'
 \frac{2q}{(1-x')^2} f(x(x'))
\end{equation}
In this case, $q$ is a choice to be made, and $x=q$ when $x'=0$. So
there are better choices for $q$ than others -- you want it to be
somewhere near where the integral is expected to reach about half its
total.

The other forms of weighting are given in the book, and may be derived
similarly. 

\section{Gaussian quadrature}

Now we have all the tools to derive one of the workhorse algorithms
for integrating functions, which is Gaussian quadrature. Gaussian
quadrature has the advantage that it yields a systematic way to write
an algorithm for integration which utilizes $N$ points, that is {\it
  exact} for any polynomial of order $2N-1$ or less. Note that this is
much better than we found before, the path we were on for Simpson's
rule, which utilized $N+1$ points to exactly integrate a polynomial of
$N$ points. It turns out that the improvement is gained by choosing
the points carefully.

We will show how to do this for the integral:
\begin{equation}
\int_{-1}^{1} {\rm d}x f(x)
\end{equation}
where $f(x)$ is a $2N-1$ degree polynomial (or less).  Clearly we can
rescale the limits as necessary above for the problem at hand.

We are seeking an exact formula for the integral of this function
which is:
\begin{equation}
\int_{-1}^{1} {\rm d}x f(x) = \sum_{i=1}^N w_i f(x_i)
\end{equation}

The derivation of this is neat. Note that the derivation in the Landau
book is extremely confusing and contains at least one error.

We will use the Legendre polynomials to aid us. In fact, it will be
the roots of the Legendre polynomials (where they are zero) that turn
out to be the locations of the integration points.

\noindent {\bf What are the Legendre Polynomials? Where have you seen
  them before.}

\begin{answer}
  They usually arise in the physics curriculum because they are the
  solutions to Laplace's Equation ($\nabla^2 \Phi =0$) under
  cylindrical symmetry. They also have some interesting properties. We
  will refer to them here as $P_n(x)$, where $-1 < x < 1$, and $n$ is
  the order of the Legendre Polynomial. They have the property that
  each Legendre Polynomial is a polynomial of order $n$:

  \begin{equation}
    P_n(x) = \sum_{i=0}^n a_i x^i 
  \end{equation}

  Specifically, they are:
  \begin{equation}
    P_n = \frac{1}{2^n n!} \frac{{\rm d}^n(x^2 -1)^n}{{\rm d} x^n}
  \end{equation}

  Or:
  \begin{eqnarray}
    P_0(x) &=& 1 \cr
    P_1(x) &=& x \cr
    P_2(x) &=& \frac{1}{2}(3x^2 -1) \cr
    P_3(x) &=& \frac{1}{2}(5x^3 - 3x) \cr
    P_4(x) &=& \frac{1}{8}(35^4 - 30x^2 + 3) \cr
    &\ldots& 
  \end{eqnarray}

  They form a complete basis set in function space (any function can be
  expressed as a sum of a sufficient number of Legendre Polynomials).
  A related property is that any polynomial of order $n$ can be
  expressed as a sum of Legendre Polynomials with orders $\le n$.
\end{answer}

\noindent {\bf All the Legendre Polynomials are orthogonal to each
  other. What does that mean?}

\begin{answer}
  It means that their dot products are zero. Functions live in a linear
  vector space. E.g. it is infinite-dimensional, and one set of basis
  functions are Dirac $\delta$-functions. You can define a dot product
  in that space as:
  \begin{equation}
    q(x) \cdot r(x) = \int_{-1}^{1} {\rm d}x q(x) r(x)
  \end{equation}
  As we will see below, this choice is not unique!

  In any case, it means that the statement that Legendre Polynomials
  are orthogonal means the following:
  \begin{equation}
    \int_{-1}^{1} {\rm d}x P_n(x) P_m(x) = \delta_{nm}
    \frac{2}{2n+1}
  \end{equation}
\end{answer}

\noindent {\bf If all the Legendre Polynomials are orthogonal to each
  other, and any polynomial of order $n$ can be expressed as a sum of
  Legendre Polynomials with order $\le n$, then what is this
  integral, for $m<n$:}
  
  \begin{equation}
    \int_{-1}^{1} {\rm d}x P_{n}(x) x^m 
  \end{equation}
  
  \begin{answer}
    0
  \end{answer}

We start by noting that $f(x)$  can be in general factored in the
following way:
\begin{equation}
f(x) = q(x) P_N(x) + r(x)
\end{equation}
where we choose $q(x)$ to be an $N-1$-degree polynomial. The first
term is therefore a polynomial of order $2N-1$, or less. Since we can
choose the coefficients of the $q(x)$ polynomial to be whatever we
want, we can always match the coefficients of all the polynomial terms
of order $N$ or greater in $f(x)$. This leaves a remainder $r(x)$
which is an $N-1$-degree polynomial (or less).

The integral:
\begin{equation}
\int_{-1}^{1} {\rm d}x q(x) P_N(x) = 0
\end{equation}
because the Legendre polynomials are always orthogonal to lower-order
polynomials.

So:
\begin{equation}
\int_{-1}^{1} {\rm d}x f(x) = \int_{-1}^{1} {\rm d}x r(x)
\end{equation}
Since $r(x)$ is an $N-1$-degree polynomial or less, there is a way to
integrate the function with $N$ points or less, as we found above.

The locations of these points can be found as follows. The integral is
now known to be writable as:
\begin{equation}
\int_{-1}^{1} {\rm d}x f(x) = \sum_{i=0}^{N-1} w_i f(x_i) =
\sum_{i=0}^{N-1} w_i q(x_i) P_N(x_i) + 
\sum_{i=0}^{N-1} w_i r(x_i)
\end{equation}
If we choose the points $x_i$ to be the $N$ roots of the Legendre
polynomial of order $N$, then:
\begin{equation}
\int_{-1}^{1} {\rm d}x f(x) = \sum_{i=0}^{N-1} w_i r(x_i)
\end{equation}
where:
\begin{equation}
r(x) = \sum_{j=0}^{N-1} \alpha_j x^j
\end{equation}
This is great, and we also know how to set the $w_i$. These are
derived in the same way as for Simpson's rule.
Using the appropriate
linear set of equations analogous to Equation \ref{eq:weights}, you
can express the solution for the coefficients $\alpha_j$ in terms of
values of the function $f(x_i)$, which by design is equal to $r(x_i)$
at the chosen points, and then given the closed form of the integral 
of the polynomial, this translates into a form for $w_i$. We can
proceed as follows.
\begin{eqnarray}
\int_{-1}^{1} {\rm d}x f(x) &=& \sum_{i=0}^{N-1} w_i r(x_i)\cr
\int_{-1}^{1} {\rm d}x r(x) &=& \sum_{i=0}^{N-1} w_i
\sum_{j=0}^{N-1} \alpha_{j} x_i^{j}\cr
\sum_{j=0}^{N-1} \frac{\alpha_j}{j!} \left[(1)^{j+1} -
(-1)^{j+1}\right] &=& \sum_{i=0}^{N-1} w_i 
\sum_{j=0}^{N-1} \alpha_{j} x_i^{j}\cr
\sum_{j=1,3,\ldots}^{N-1} \frac{2 \alpha_j}{j!} &=&
\sum_{{j}=0}^{N-1}
\alpha_{j}
\sum_{i=0}^{N-1} w_i 
x_i^{j}
\end{eqnarray}
Since this has to hold for {\it all} choices of $\alpha_j$, this leads
to a set of $N$ equations for the $N$ unknown $w_i$ values, from which
the $w_i$ can be determined. As it happens, the set of $w_i$ that
solve this set of equations are:
\begin{equation}
w_i = \frac{2}{\left(1-x_i^2\right) \left(P_n'(x_i)\right)^2}
\end{equation}

The notebook shows an implementation given the weights and locations
determined for $N=4$, and demonstrates performance up to $2N-1$.

Rather than determine weights and locations yourself to higher order,
the SciPy routines in its {\tt integrate} module already contain this
information. In particular, {\tt fixed\_quad} performs the fixed-order
Gaussian quadrature that we find here.

\section{Generalizations of Gaussian quadrature}

The Gaussian quadrature method is good for smooth functions. However,
if it is not smooth, or in particular has singularities, it will be an
issue. In addition, there turns out to be plenty of scope in
generalizing the method to handle certain non-polynomial functions
exactly.

Specifically, it turns out that you can generally find {\it exact}
expressions for integrals of the following form:
\begin{equation}
\int_{-1}^{1} {\rm d}x W(x) f(x) 
\end{equation}
where $W(x)$ is a known function and $f(x)$ is a polynomial. 

The proof involves redefining the dot product between two functions
$q(x)$ and $r(x)$:
\begin{equation}
  q(x) \cdot r(x) = \int_{-1}^{1} {\rm d}x W(x) q(x) r(x)
\end{equation}
Then it turns out we can find a complete basis set of polynomals that
are orthogonal under this definition. Legendre Polynomials are just
one case of this, for $W(x)=1$. The locations are determined by the
roots of these new polynomials, and the weights are determined in an
analogous manner to the Gauss-Legendre case.

As an example, look at the problem:
\begin{equation}
\int_{-1}^{1} {\rm d}x \frac{1}{\sqrt{1-x^2}} = \pi
\end{equation}
If we use regular Gaussian quadrature, our answers are very bad.

But we can define $W(x)=1/\sqrt{1-x^2}$ and $f(x)=1$. This is called
{\it Gauss-Chebyshev} quadrature. SciPy doesn't have this directly,
but it does have a routine that gives you the roots and weights.  This
works very well for integrating over this singularity.

Another generally useful form has $W(x) = \exp(-x^2)$. This yields
Gauss-Hermite polynomials, and Gauss-Hermite quadrature. A slightly
altered Gaussian is a common form for a number of real-world
distributions. 

\section{A physical example: nuclear reaction rates}

One example of a process that involves an integral that needs to be
estimated numerically is that of nuclear reactions in stars. 

Nuclear fusion reactions in stars are driven by the following
process. The center of the star consists of very hot ionized gas. The
nuclei of the atoms in the gas have high enough energies that two of
them can get close enough to one another to tunnel through their
Coulomb repulsion into their energetically preferred bound state (or
to otherwise interact). The rates of these reactions are driven by the
number densities of the nuclear species, their temperatures, and their
Coulomb charges. 

Specifically, the cross-section for the reactions can be written:
\begin{equation}
\sigma(E) = \frac{S(E)}{E} \exp\left[-(E_c / E)^{1/2}\right]
\end{equation}
where we will set the slowly varying function $S(E)=1$ for simplicity
and:
\begin{equation}
E_c = \frac{2\pi^2 Z_1^2 Z_2^2 e^4 \mu}{\hbar^2}
\end{equation}

Then the reaction rate per unit volume, which ultimately determines in
the case of our Sun how much energy is produced, and thus how brightly
it shines, is:
\begin{equation}
r_{12} = \frac{n_1 n_2}{(\pi \mu)^{1/2}}
\left(\frac{2}{kT}\right)^{3/2}  \int_0^\infty {\rm d}E S(E)
\exp\left[-(E_c/E)^{1/2}\right] \exp\left(-E/kT\right)
\end{equation}
Basically, this is an integral over all of the relative kinetic
energies the particles can have, weighted by the cross section of
interaction.
This governs the rates for any two species, but of course in reality
the Sun's core contains many different simultaneous reactions. We will
just calculate a small part of this problem.

We may be interested in a very important dependence, which is the
dependence of this reaction rate on temperature. This dependence is
part of what sets the overall structure of stars. 

For two protons, the value:
\begin{equation}
E_c = \frac{2 \pi^2 e^4 m_p}{2 \hbar^2} \approx 7.9 \times 10^{-14}
\mathrm{\quad kg~m}^2\mathrm{~s}^{-2}
\end{equation}

The temperatures are of order $10^7$ K, so:
\begin{eqnarray}
  kT &\sim&
  1.3806 \times 10^{-23} \times 10^{7} \mathrm{\quad kg~m}^2\mathrm{~s}^{-2}\cr
&\sim& 
  1.3806 \times 10^{-16} \mathrm{\quad kg~m}^2\mathrm{~s}^{-2}
\end{eqnarray}
So typically $E_c/kT \sim 1000$.

For convenience and to keep us as safe as possible from underflows and
overflows, we want to integrate over variables that are closer to
unity, not $10^{-15}$. Also, if we look at the integrand it is clear
that $E_c$ and $kT$ do not matter individually (except as overall
scale factors), just their ratio; we can avoid doing some integrals if
we cast results just in terms of the ratio.  So it makes sense to
perform the transformation:
\begin{eqnarray}
 x &=& E / kT \cr
 {\rm d}x &=& {\rm d}E / kT
\end{eqnarray}
and then define $R = E_c /kT$,
which then gives us:
\begin{eqnarray}
r_{12} &=& \frac{2 n_1 n_2}{(\pi \mu)^{1/2}}
\left(\frac{2}{kT}\right)^{1/2}
\int_0^\infty {\rm d}x S(E(x))
\exp\left[-(R/x)^{1/2}\right] \exp\left(- x \right) \cr
&=& \frac{2 n_1 n_2}{(\pi \mu)^{1/2}}
\left(\frac{2}{kT}\right)^{1/2} I(R)
\end{eqnarray}
where
\begin{equation}
I(R) = \int_0^\infty {\rm d}x S(E(x))
\exp\left[-(R/x)^{1/2}\right] \exp\left(- x \right) 
\end{equation}
Now, the integration involves just $R$, and the rest of the scaling of
the reaction rate is just multiplication.

The integral has the form:
\begin{equation}
I(R) = \int_0^{\infty} {\rm d}x f(x; R) \exp(-x)
\end{equation}
where
\begin{equation}
f(x; R) = 
\exp\left[-(R/x)^{1/2}\right]
\end{equation}
This is amenable to a form of Gaussian quadrature, specifically
Gauss-Laguerre quadrature. This means:
\begin{equation}
I(R) \approx \sum_{i=1}^N w_i f(x_i; R)
\end{equation}

If we want to know how $r_{12}$ scales with temperature:
\begin{equation}
r_{12} \propto R^{1/2} I(R)
\end{equation}

\section{Monte Carlo}

% What about quasi random?

Another way to estimate integrals is the Monte Carlo technique, which
is useful especially for multidimensional integrals, but we introduce
in the 1D case.

The basic idea here is to distribute a bunch of random $x$ between $a$
and $b$. Then to issue random $y$ values between the minimum and
maximum of $f(x)$ (or any values more extreme than those). The
estimate then becomes:
\begin{equation}
\int_a^b {\rm d}x f(x) \approx (b-a) (f_{\rm max} - f_{\rm min})
\frac{N(y<f(x))}{N}
\end{equation}

\noindent {\bf Can you guess what the error in this estimate is?}

\begin{answer}
It will scale as $\sqrt{N(y<f(x))}$, because this is a Poisson-like
process.  Specifically:
\begin{equation}
\langle\delta I^2 \rangle^{1/2} \approx (b-a) (f_{\rm max} - f_{\rm min})
\frac{\sqrt{N(y<f(x))}}{N}
\end{equation}
and so
\begin{equation}
\frac{\langle\delta I^2 \rangle^{1/2}}{I} \approx 
\frac{1}{\sqrt{N(y<f(x))}}
\end{equation}
This means you need a {\it ton} of points to get good
precision. Getting to a part in $10^{10}$ would require $10^{20}$
points. So this isn't a good strategy in 1-dimension! In fact, there
are better methods in higher dimensions too.
\end{answer}

The better method is {\it mean value integration}. This uses the fact
that in some range $a<x<b$:
\begin{equation}
\langle f(x) \rangle = \frac{\int_a^b \dd{x}  f(x)}{b-a}
\end{equation}
So if we estimate:
\begin{equation}
\langle f(x) \rangle \approx \frac{1}{N}\sum_{i=1}^N f(x_i)
\end{equation}
for uniformly, randomly chosen $a<x_i<b$, we can estimate the integral
as:
\begin{equation}
I = \int_a^b \dd{x}  f(x) \approx (b-a) \langle f(x)\rangle
\end{equation}

The error in $I$ also scales as $1/\sqrt{N}$ as for the stone-throwing
case, but you get to use all of the points.  Specifically:
\begin{equation}
\langle \delta I^2 \rangle = \frac{\sigma_f^2}{N} 
\end{equation}
where:
\begin{equation}
\sigma_f^2 = \int_a^b \dd{x} (f(x) - \langle f\rangle)^2
\end{equation}

\section{Multidimensional integrals}

The real value of the Monte Carlo method is in
multidimensions. Traditional methods have difficulty in
multidimensions. In the general case in 2D:
\begin{equation}
I = \int \dd{x} \int \dd{y} f(x, y)
\end{equation}
one case in principle construct function that evaluates using some
method:
\begin{equation}
I_y(x) = \int \dd{y} f(x, y)
\end{equation}
and then build that into an outer integral:
\begin{equation}
I = \int \dd{x} I_y(x)
\end{equation}
But if there are (e.g.) 100 evaluations of $I_y(x)$, and each of those
requires 100 evaluations of $f(x,y)$, that is 10000 evaluations. That
is, the number of evaluations goes as $N^d$, where $d$ is the number
of dimensions. 

It is very important to not blindly evaluate multi-dimensional
integrals if you do not have to! E.g. if the function is separable:
\begin{equation}
f(x, y) = f_x(x) f_y(y)
\end{equation}
obviously you can avoid 2D integrals entirely. What is not always as
obvious are cases where you can change variables to avoid an inner
integral or simplify it. This is not obviously separable:
\begin{equation}
  I = \int_0^\infty \dd{x} \int_0^\infty \dd{y} f(x, y) =
  \int_0^\infty \dd{x} \int_0^\infty \dd{y} f(xy) =
  \int_0^\infty \dd{x} \int_0^\infty \dd{y} \exp(- xy) \sin^3(xy)
\end{equation}
but if I change variables:
\begin{eqnarray}
y' &=& xy \cr
\dd{y'} &=& x\dd{y}
\end{eqnarray}
then it is
\begin{equation}
  I = \int_0^\infty \dd{x} \int_0^\infty \dd{y} f(xy) =
  \int_0^\infty \dd{x} \frac{1}{x} \int_0^\infty \dd{y'} f(y') =
  \left(\int_0^\infty \dd{x} \frac{1}{x}\right)
  \left(\int_0^\infty \dd{y'} f(y') \right)
\end{equation}
Note how this required the change of variables to not affect the
limits of integration!! The point is, before you do the integral
numerically do as much math as you can on it!

But in those cases where you cannot avoid multidimensional integrals
of high dimension (often three and certainly higher) Monte Carlo
methods are the way to go.

\noindent {\bf How do you use the mean value integration method in
  multidimensions?}

\begin{answer}
The mean value method works fine in such cases; you just need to draw
points in the $N$-dimensional space being integrated.
\end{answer}

There are two useful techniques to improve the MC method for
integration. The first is {\it variance reduction}. If we have some
function $g(x)$ which is similar to $f(x)$ but can be integrated
analytically, then it is useful to calculate:
\begin{equation}
  I = \int \dd^N{\vec{x}} f(\vec{x})
  = \int \dd^N{\vec{x}} (f(\vec{x}) - g(\vec{x})) + 
  \int \dd^N{\vec{x}} g(\vec{x})
\end{equation}
where the first term is performed with MC, and the second term is
analytic. 

\noindent {\bf Explain why this method reduces the error.}

\begin{answer}
Because the error is related to $\sigma_f$. If you perform an
integration over $f-g$, and $\sigma_{f-g}$ is smaller than $\sigma_f$,
then you are reducing the random error.
\end{answer}

A similar method of reducing the error is {\it importance
  sampling}. This is slightly different. You use a random distribution
with a probability $P(x)$ (not uniform). Then you note:
\begin{equation}
  I = \int \dd^N{\vec{x}} f(\vec{x}) = 
  \int \dd^N{\vec{x}} P(\vec{x}) \frac{f(\vec{x})}{P(\vec{x})}
\end{equation}
If you have $N$ random points $x_i$ distributed as $P(x)$, then:
\begin{equation}
 I \approx \frac{1}{N} \sum_{i}  \frac{f(\vec{x})}{P(\vec{x})}
\end{equation}

If $f(x)/P(x)$  is more uniform than $f(x)$ then the error in the
result is smaller, for the same reason as in the variance reduction
method. 

An example of either case is a function $f(x)$ which is close to
$\exp(-x^2)$.

