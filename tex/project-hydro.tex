\documentclass[11pt, preprint]{aastex}

\usepackage{environ}
\usepackage{xcolor}
\usepackage{hyperref}
\usepackage{rotating}  
\usepackage{amsmath}

\newcommand{\todo}[1]{{\bf #1}}
\newcommand{\dd}[1]{\ensuremath{{\rm d}#1}}
\newcommand{\mat}[1]{\ensuremath{{\bf #1}}}
\DeclareMathOperator{\sech}{sech}

\newif\ifanswers

\NewEnviron{answer}{\ifanswers\color{blue}\expandafter\BODY\fi}

\newenvironment{ditemize}
{ \begin{list}{}{%
\setlength{\topsep}{0pt}% 
\setlength{\partopsep}{3pt}% 
\setlength{\itemsep}{1pt}\setlength{\parsep}{1pt}% 
\setlength{\itemindent}{0pt}\setlength{\listparindent}{12pt}%
\setlength{\leftmargin}{24pt}\setlength{\rightmargin}{0in}%
\setlength{\labelsep}{6pt}\setlength{\labelwidth}{6pt}%
\renewcommand{\makelabel}{\makebox[\labelwidth][l]{$\bullet$\hspace{\fill}}}}}
{\end{list}}


\begin{document}

\title{\bf Computational Physics Project / Hydrodynamics Code}

This project is largely copied from Andrew MacFadyen's ``How To Write
A Hydrodynamics Code.'' It walks through the implementation of a 1D
and 2D hydrodynamics code.

The equations of one-dimensional hydrodynamics can be
written in the conservation form as:
\begin{equation}
\label{eq:conservation}
  \frac{\partial \vec{U}}{\partial t}
  + \frac{\partial \vec{F}}{\partial x} = 0
\end{equation}
where $\vec{U} = \{ \rho, \rho v, E\}$ are the conserved variables for
mass, momentum, and energy, and $\vec{F} = \{ \rho v, \rho v^2 + P, (E
+ P) v\}$ are the fluxes of those quantities. $\rho$ is the density,
$v$ is the velocity, $P$ is the pressure, and $E=\rho e +
\frac{1}{2}\rho v^2$ is the total energy density. The quantity $e$ is
the specific internal energy (i.e. internal energy per unit mass).

To close the equations requires an equation of state $P=P(\rho,
e)$. For an ideal gas:
\begin{equation}
\label{eq:eos}
P = (\gamma - 1)\rho e,
\end{equation}
where $\gamma$ is the adiabitic index.

\section{Low-order 1D Method}

If we establish a 1D grid of cells along the axis $x$, whose centers
are indexed by $i$ and whose sizes are uniform with size $\Delta x$,
and we know the fluxes at the interfaces of the cells at some time,
the time derivative for $\vec{U}$ can be calculated discretely:
\begin{equation}
  \frac{\partial \vec{U}}{\partial t} = L\left(\vec{U}\right)
  = - \frac{\vec{F}_{i+1/2} - \vec{F}_{i-1/2}}{\Delta x},
\end{equation}
where $i+1/2$ refers to the interface between $i$ and $i+1$ and
similarly for $i-1/2$.

Then with first-order forward Euler we can evolve the conserved
quantities:
\begin{equation}
\vec{U}(t_{n+1}) = \vec{U}(t_n) + \Delta t L\left(\vec{U}(t_n)\right),
\end{equation}
where $\Delta t$ is the time step.

We need to calculate the fluxes given the conditions in neighboring
cells. We could interpolate the densities, velocities, and energies to
calculate the fluxes, but this turns out not to be the best path
forward. Instead, most work uses a variant of the Godunov method,
which treats each interface as a little ``Riemann'' problem describing
what happens at the interface of a 1D discontinuity.  That is, imagine
that the fluid literally had piecewise constant values of its
properties; at the interfaces, the changes of density, pressure, etc,
would drive waves. These little ``shock tube'' problems can be solved
and the flow of density, momentum density, and energy density in these
waves can be calculated, and then used. This method turns out to be
more accurate.

The exact solution to the Riemann problem is very expensive
computationally; that is, too expensive to calculate for every cell in
a simulation. But there are approximate solutions in common use. The
Harten-Lax-van Leer (HLL) approximation is a classic efficient one:
\begin{equation}
\vec{F}^{\rm HLL} = \frac{\alpha^{+}\vec{F}^L + \alpha^{-} \vec{F}^R -
  \alpha^{+} \alpha^{-} \left(\vec{U}^R - \vec{U}^L\right)}{\alpha^{+}
  + \alpha^{-}}.
\end{equation}
where $L$ and $R$ refer to the ``left'' and ''right'' states ($i$ and
$i+1$, for example). The $\alpha^\pm$ values are given by:
\begin{equation}
  \alpha^\pm = {\tt max}\left\{0,
  \pm \lambda^{\pm}\left(\vec{U}^L\right),
  \pm \lambda^{\pm}\left(\vec{U}^R\right)\right\}
\end{equation}
where
\begin{eqnarray}
  \lambda^{\pm} &=& v\pm c_s \cr
  c_s &=& \sqrt{\frac{\gamma P}{\rho}}
\end{eqnarray}
It isn't trivial to say why this works; it is tuned approximation
known to have nice properties. You might want to think of the
velocities $\alpha^\pm$ as expressing something about the velocities
of the waves emanating from the initial discontinuity between the two
cells.

In implementing this method, you do not want one the waves to travel
further than one cell size over the course of a time step. This is the
Courant-Friedrich-Levy condition. Therefore the time step is
constrained by:
\begin{equation}
\Delta t < \Delta x / {\tt max}\left(\alpha^\pm\right),
\end{equation}
which you should note needs to hold everywhere on the grid.

\section{Prep Work}

\begin{itemize}
\item Rescale the equations \ref{eq:conservation} and \ref{eq:eos} to
  a set of unitless variables and to the minimum number of
  dimensionful parameters that you can. You should perform your
  numerical analysis in these variables, and rescale the solutions to
  physical variables at the end.
\item Consider a set of unit tests that you will implement to test
  this code. An example might be a test of your {\tt minmod} function,
  but consider other components of the code that will benefit from
  unit tests.
\end{itemize}

\section{Sod Shock Tube Problem}

An initial test of the code is to run a Sod shock tube problem. Note
that this is basically solving on a grid the same Riemann problem that
is approximated at the cell level. Perform the following with the
low order code.

Set up your Python code so that you separate the initialization from
the evolution. You should have a script which imports three modules: 
\begin{itemize}
  \item Initialization (with functions setting up an initial grid
    given parameters).
  \item Evolution (integrates the equations and returns a history of
    the evolution of the grid).
  \item Output (writes the output to a file or files).
\end{itemize}
Then {\it in a separate script} you should be able to input the final
files and make plots (again, a very good idea would be to separate the
functionality into two different modules).

\begin{itemize}
\item Initialize two halves of the tube (left and right) to different
  states, where:
  \begin{eqnarray}
    \frac{P_L}{P_R} &=& 8 \cr
    \frac{\rho_L}{\rho_R} &=& 10
  \end{eqnarray}
  and the velocity is zero. Set $\gamma=1.4$. Plot and examine the
  density, pressure, and velocity. Exchange left and right and make
  sure the answer is exactly the same both ways.
\item This problem can be solved exactly. Compare your answer to the
  exact answer
  \href{https://cococubed.com/code\_pages/exact\_riemann.shtml}{using
    the code here}. This requires compiling a piece of Fortran, and a
  minor change in the parameters near its beginning. Test several
  initial parameter settings.
\item Test the convergence of the method by doubling and quadrupling
  the number of spatial grid points (make sure you handle the time
  steps consistently so as not to violate the Courant-Friedrich-Levy
  condition).
\end{itemize}

\section{Higher Order 1D Method}

We can reduce the approximation error by working at higher order
in space and time.

In time we can use a version of third-order Runge-Kutta due to Shu \%
Osher. This method starts with $\vec{U}(t_n)$ and advances to
$\vec{U}(t_{n+1})$ as follows:
\begin{eqnarray}
\vec{U}^{(1)} &=& \vec{U}(t_n) + \Delta t L\left(\vec{U}(t_n)\right)
\cr
\vec{U}^{(2)} &=& \frac{3}{4} \vec{U}(t_n) + \frac{1}{4} \vec{U}^{(1)}
+ \frac{1}{4} \Delta t L\left(\vec{U}(t_n)\right)\cr
\vec{U}(t_{n+1}) &=& \frac{1}{3} \vec{U}(t_n) + \frac{2}{3} \vec{U}^{(2)}
+ \frac{2}{3} \Delta t L\left(\vec{U}(t_n)\right)
\end{eqnarray}

In space we go to higher order by using more spatial points to
determine the left and right conditions on each cell interface from
which to calculate fluxes, instead of just the piecewise constant
model. We cannot use a fully self-consistent interpolation, since that
does not lead to a discontinuity! So the left conditions are
calculated using a left-biased set of points, and the right conditions
from a right-biased set of points.

Specifically, for $c= \rho$, $P$, and $v$ (not the components of
$\vec{U}$!):
\begin{equation}
  c_{i+1/2}^L = c_i + \frac{1}{2}
  {\tt minmod}\left[\theta \left(c_i - c_{i-1}\right),
  \frac{1}{2}\left(c_{i+1} - c_{i-1}\right), 
  \theta \left(c_{i+1} - c_{i}\right) \right] 
\end{equation}
where $\theta$ is a parameter we can choose between 1 and 2, and:
\begin{equation}
  {\tt minmod}(x, y, z) = \frac{1}{4}\left| {\rm sgn}(x) + {\rm
    sgn}(y) \right| \,
  \left( {\rm sgn}(x) + {\rm sgn}(z)\right) {\rm min}\left(|x|, |y|,
  |z|\right)
\end{equation}
For the right interface:
\begin{equation}
  c_{i+1/2}^R = c_{i+1} + \frac{1}{2}
  {\tt minmod}\left[\theta \left(c_{i+1} - c_{i}\right),
  \frac{1}{2}\left(c_{i+2} - c_{i}\right), 
  \theta \left(c_{i+2} - c_{i+1}\right) \right] 
\end{equation}

\begin{itemize}
\item Implement this higher order implementation. Start it with the
  same initial conditions as above.
\item Repeat the test against a known exact solution and repeat the
  convergence test. Do you see the expected improvement in the
  approximation error and its convergence properties?
\item For at least three choices of grid size, perform a timing test
  using both the low order and high order cases, and compare their
  speed. Is the higher order faster than lower order at fixed {\it
    computation time}?
\end{itemize}

\section{2D Higher-order Method}

Extending to higher number of spatial dimensions is straightforward
for a simple regular mesh. For example, in two dimensions:
\begin{equation}
\label{eq:conservation}
  \frac{\partial \vec{U}}{\partial t}
  + \frac{\partial \vec{F}}{\partial x}
  + \frac{\partial \vec{G}}{\partial y} = 0
\end{equation}
where in this case the fluxes and conserved quantities are
generalized:
\begin{eqnarray}
\vec{U} &=& \left\{\rho, \rho v_x, \rho v_y, E\right\} \cr
\vec{F} &=& \left\{\rho v_x, \rho v_x^2 + P, \rho v_x v_y, (E+P) v_x\right\} \cr
\vec{G} &=& \left\{\rho v_y, \rho v_x v_y, \rho v_y^2 + P, (E+P) v_y\right\}
\end{eqnarray}
and the total energy is $E=\rho e + \frac{1}{2}\rho(v_x^2 + v_y^2)$.

In this case, the time derivative of the consierved quantities are:
\begin{equation}
  \frac{\partial \vec{U}}{\partial t} = L\left(\vec{U}\right)
  = - \frac{\vec{F}_{i+1/2} - \vec{F}_{i-1/2}}{\Delta x}
    - \frac{\vec{G}_{i+1/2} - \vec{G}_{i-1/2}}{\Delta x}
\end{equation}

The time steps can again be done as a third-order Runge-Kutta, and the
high order version of the flux determinations can be performed in an
analagous way as above.

You shoud implement this 2D case. The boundary conditions will matter
a little more in this case, so be careful of that.

\begin{itemize}
\item Implement periodic boundary conditions in the $y$-direction, and
  try to implement the 1D Sod shock tube problem in the
  $x$-direction. Evaluate whether the system is stable.
\item Implement periodic boundary conditions in both directions, and
  set initial conditions such that the top part of the fluid (high
  $y$) is moving relative to the bottom part (low $y$), and otherwise
  the fluids are identical. Examine how the fluid behaves under
  different choices of the relative velocity.
\item Near the center of the grid (say about $1/10$ of the radius),
  put a region with a density 100 times the rest of the grid. This
  should drive a blast wave outwards.
\end{itemize}

\section{Bonus: Other tests}

Consult
\href{https://www.astro.princeton.edu/~jstone/Athena/tests}{Jim
  Stone's web page} for some 1D and 2D tests of potential
interest. Consider the Rayleigh-Taylor or the Double Mach reflection
tests. Note both require a special boundary condition. 

\end{document}
