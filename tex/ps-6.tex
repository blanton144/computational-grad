\documentclass[11pt, preprint]{aastex}

\usepackage{environ}
\usepackage{xcolor}
\usepackage{hyperref}
\usepackage{rotating}  
\usepackage{amsmath}

\newcommand{\todo}[1]{{\bf #1}}
\newcommand{\dd}[1]{\ensuremath{{\rm d}#1}}
\newcommand{\mat}[1]{\ensuremath{{\bf #1}}}
\DeclareMathOperator{\sech}{sech}

\newif\ifanswers

\NewEnviron{answer}{\ifanswers\color{blue}\expandafter\BODY\fi}

\newenvironment{ditemize}
{ \begin{list}{}{%
\setlength{\topsep}{0pt}% 
\setlength{\partopsep}{3pt}% 
\setlength{\itemsep}{1pt}\setlength{\parsep}{1pt}% 
\setlength{\itemindent}{0pt}\setlength{\listparindent}{12pt}%
\setlength{\leftmargin}{24pt}\setlength{\rightmargin}{0in}%
\setlength{\labelsep}{6pt}\setlength{\labelwidth}{6pt}%
\renewcommand{\makelabel}{\makebox[\labelwidth][l]{$\bullet$\hspace{\fill}}}}}
{\end{list}}


\begin{document}

\title{\bf Computational Physics / PHYS-GA 2000 / Problem Set \#6
\\ Due October 31, 2023 }

You {\it must} label all axes of all plots, including giving the {\it
  units}!!

\begin{enumerate}
%\item Calculate derivative of the function:
  %\begin{equation}
    %f(x) = \frac{1}{\sqrt{2\pi}} \exp\left(-x^2/2\right)
  %\end{equation}
  %three ways: (a) analytically; (b) with second-order finite
  %difference; and (c) with {\tt autodiff} using the {\tt jax}
  %implementation. Compare the results, and also compare the speeds.
\item Here we will perform a simple Principal Components Analysis on a
  real data set. Note this problem is not as daunting as it might look
  at first; there are many words because I am walking you through all
  the steps one by one.  Download
  \href{https://www.dropbox.com/scl/fi/9cmawl762qtuwiouhmafs/specgrid.fits?rlkey=d2zcco18t5vfbib5nuxmsu411&dl=0}{this
    file} which contains the central optical spectra of 9,713 nearby
  galaxies from the Sloan Digital Sky Survey (note that although this
  file is 150 Mb, this is a small sample!  larger data sets exist of
  millions, though they are lower quality). I have done some the work
  for you by interpolating all of the spectra onto the same restframe
  wavelength grid. Now do the following:
  \begin{enumerate}
    \item Read the data in using the {\tt astropy} package, in
      particular using {\tt astropy.io.fits}. This data set is a
      special format called the Flexible Image Transport System (FITS)
      format, common in astronomy. Its only virtue is that it is a
      standard in astronomy. You should be able to {\tt pip install
        astropy}, and then:
      \begin{verbatim}
        hdu_list = astropy.io.fits.open('specgrid.fits')
        logwave = hdu_list['LOGWAVE'].data
        flux = hdu_list['FLUX'].data
      \end{verbatim}
      {\tt logwave} will be $\log_{10}\lambda$ for $\lambda$ in
      Angstroms. {\tt flux} will be in $10^{-17}$ erg s$^{-1}$
      cm$^{-2}$ \AA$^{-1}$, and is the spectrum. Plot a handful of the
      galaxies. Being a physicist and knowing something about the
      transitions in the Hydrogen atom, do you notice any features
      that relate to it?
    \item There are two processing steps that will make the PCA more
      meaningful. First, all of these galaxies are at different
      distances, so their fluxes span a large dynamic range; so first
      normalize all the fluxes so their integrals over wavelength are
      the same. It is close enough to sum over the values to estimate
      the integral (i.e. don't worry about the fact that
      $\Delta\lambda$ varies across the grid). Keep each normalization
      available.
     \item Second, the mean flux at every wavelength is positive; this
       will mean that a PCA will spend an eigenvector to explain the
       mean offset from zero. Instead, we will first subtract off the
       mean $\vec{f}_m$ of the normalized spectra. This will leave
       residuals $\vec{r}_i = \vec{f}_i - \vec{f}_m$ of all the
       galaxies $i$ varying around zero. Keep the mean spectrum
       available.
     \item Now perform the PCA. The idea of the PCA is to find the
       eigenvectors of the covariance matrix of the distribution. This
       covariance matrix {\it can} be calculated as follows:
       \begin{equation}
         \mat{C} = \frac{1}{N_{\rm gal}} \sum_{ij} \vec{r}_i \vec{r}_j
       \end{equation}
       where $i$ and $j$ index the galaxies. If I recast the residuals
       as a matrix $R_{ij}$ this is $\mat{C} =
       \mat{R}\cdot\mat{R}^T$. So construct this matrix (it should be
       $N_{\rm wave}\times N_{\rm wave}$), and find its
       eigenvectors. Make a plot of the first five eigenvectors.
     \item It is also possible to find these eigenvectors directly
       from $\mat{R}$ using SVD. Consider the linear problem, which
       finds a set of coefficients $\vec{x}$ to multiply the given
       spectra by, to explain some spectrum $\vec{f}$:
       \begin{equation}
         \mat{R}\cdot\vec{x} = \vec{f}
       \end{equation}
       We know the SVD decomposition of \mat{R} yields a rotation
       $\mat{V}$ into the space where the covariance matrix of the
       uncertainties in $\vec{x}$ is diagonal. This covariance matrix
       is $\mat{R}^T\cdot\mat{R}$. Another way to see this is that the
       covariance can be written:
       \begin{equation}
         \mat{R}^T\cdot\mat{R} =
         \mat{V}\cdot\mat{W}\cdot\mat{U}^T\cdot\mat{U}\cdot\mat{W}\cdot{V}^T
       \end{equation}
       The central structure is a diagonal matrix, and $\mat{V}$ is
       unitary (its inverse is its transpose), so the matrix $\mat{V}$
       is composed of the right eigenvectors of
       $\mat{R}^T\cdot\mat{R}$ (with eigenvalues which are the square
       of the singular values). So now find the eigenvectors using an
       SVD decomposition of $\mat{R}$ and show that the vectors are
       equivalent to what you found before. Compare the spend of this
       to the method in the previous bullet.
    \item Can you think of reasons you might want to use SVD instead
      of constructing the covariance matrix and finding its
      eigenvectors? How does the condition number of $\mat{C}$ compare
      to that of $\mat{R}$?
     \item A common use case for PCA is to have components that can
       compactly approximate data. This can be a form of compression
       (e.g. storing only the first few PCA coefficients of a spectrum
       instead of the full spectrum) though it can also be useful in
       fitting models (e.g. the redshift-fitting algorithm of SDSS
       uses a PCA basis fit at each redshift and minimizes $\chi^2$
       over redshift). Each original input spectrum can be expressed
       as the mean spectrum, plus the full set of coefficients $c_i$
       multiplying the eigenspectra, times the original
       normalization. But you can see what happens if you keep the
       first $N_c=5$ coefficients. Create approximate spectra based on
       keeping only the first five coefficients. Note that if you
       think about the math, you don't need to perform {\it another}
       matrix inversion or decomposition to find the coefficients, you
       just have to rotate the spectra into the eigenspectrum basis.
     \item Plot $c_0$ vs $c_1$ and $c_0$ vs $c_2$.
     \item Now try varying $N_c =1,2,\ldots, 20$ and calculate the
       squared fractional residuals between the spectra and the
       reconstituted, approximate spectra as a function of $N_c$. You
       should see the squared residuals declining. How small is the
       fractional error for $N_c =20$?
  \end{enumerate}
\end{enumerate}

\end{document}
