\documentclass[11pt, preprint]{aastex}

\usepackage{environ}
\usepackage{xcolor}
\usepackage{hyperref}
\usepackage{rotating}  
\usepackage{amsmath}

\newcommand{\todo}[1]{{\bf #1}}
\newcommand{\dd}[1]{\ensuremath{{\rm d}#1}}
\newcommand{\mat}[1]{\ensuremath{{\bf #1}}}
\DeclareMathOperator{\sech}{sech}

\newif\ifanswers

\NewEnviron{answer}{\ifanswers\color{blue}\expandafter\BODY\fi}

\newenvironment{ditemize}
{ \begin{list}{}{%
\setlength{\topsep}{0pt}% 
\setlength{\partopsep}{3pt}% 
\setlength{\itemsep}{1pt}\setlength{\parsep}{1pt}% 
\setlength{\itemindent}{0pt}\setlength{\listparindent}{12pt}%
\setlength{\leftmargin}{24pt}\setlength{\rightmargin}{0in}%
\setlength{\labelsep}{6pt}\setlength{\labelwidth}{6pt}%
\renewcommand{\makelabel}{\makebox[\labelwidth][l]{$\bullet$\hspace{\fill}}}}}
{\end{list}}


\begin{document}

\title{\bf Computational Physics / PHYS-GA 2000 / Problem Set \#6
\\ Due October 31, 2023 }

You {\it must} label all axes of all plots, including giving the {\it
  units}!!

\begin{enumerate}
\item Calculate derivative of the function:
  \begin{equation}
    f(x) = \frac{1}{\sqrt{2\pi}} \exp\left(-x^2/2\right)
  \end{equation}
  three ways: (a) analytically; (b) with second-order finite
  difference; and (c) with {\tt autodiff} using the {\tt jax}
  implementation. Compare the results, and also compare the speeds.
\item Here we will perform a simple Principal Components Analysis on a
  real data set. Download \href{}{this file} which contains the
  central optical spectra of 9,713 nearby galaxies from the Sloan
  Digital Sky Survey (note that although this file is 150 Mb, this is
  a small sample!  larger data sets exist of millions, though they are
  lower quality). I have done some the work for you by interpolating
  all of the spectra onto the same restframe wavelength grid. Now do
  the following:
  \begin{enumerate}
    \item Read the data in using the {\tt astropy} package, in
      particular using {\tt astropy.io.fits}. This data set is a
      special format called the Flexible Image Transport System (FITS)
      format, common in astronomy. Its only virtue is that it is a
      standard in astronomy. You should be able to {\tt pip install
        astropy}, and then:
      \begin{verbatim}
        hdu_list = astropy.io.fits.open('specgrid.fits')
        logwave = hdu_list['LOGWAVE'].data
        flux = hdu_list['LOGWAVE'].flux
      \end{verbatim}
      {\tt logwave} will be $\log_{10}\lambda$ for $\lambda$ in
      Angstroms. {\tt flux} will be in $10^{-17}$ erg s$^{-1}$
      cm$^{-2}$ \AA$^{-1}$, and is the spectrum. Plot a handful
      of the galaxies.
    \item There are two processing steps that will make the PCA more
      meaningful. First, all of these galaxies are at different
      distances, so their fluxes span a large dynamic range; so first
      normalize all the fluxes so their integrals over wavelength are
      the same.
     \item Second, the mean flux at every wavelength is positive; this
       will mean that a PCA will spend an eigenvector to explain the
       mean offset from zero. Instead, we will first subtract off the
       mean $\vec{f}_m$. This will leave residuals $\vec{r}_i =
       \vec{f}_i - \vec{f}_m$ of all the galaxies $i$ varying around
       zero.
     \item Now perform the PCA. The idea of the PCA is to find the
       eigenvectors of the covariance matrix of the distribution. This
       covariance matrix {\it can} be calculated as follows:
       \begin{equation}
         \mat{C} = \frac{1}{N_{\rm gal}} \sum_{ij} \vec{r}_i \vec{r}_j
       \end{equation}
       where $i$ and $j$ index the galaxies. If I recast the residuals
       as a matrix $R_{ij}$ this is $\mat{R}\cdot\mat{R}^T$. So
       construct this matrix (it should be $N_{\rm wave}\times N_{\rm
         wave}$, and find its eigenvectors. Make a plot of the first
       five eigenvectors.
     \item It is also possible to find these eigenvectors directly
       from $\mat{R}$ using SVD. Consider the linear problem, which
       finds a set of coefficients $\vec{x}$ to multiply the given
       spectra by, to explain some spectrum $\vec{f}$:
       \begin{equation}
         \mat{R}\cdot\vec{x} = \vec{f}
       \end{equation}
       We know the SVD decomposition of \mat{R} yields a rotation
       $\mat{V}$ into the space where the covariance matrix of the
       uncertainties in $\vec{x}$ is diagonal. This covariance matrix
       is $\mat{R}^T\cdot\mat{R}$. So now find the eigenvectors using
       an SVD decomposition of $\mat{R}$ and show that the vectors are
       equivalent to what you found before.
  \end{enumerate}
\end{enumerate}

\end{document}
