\documentclass[11pt, preprint]{aastex}

\usepackage{environ}
\usepackage{xcolor}
\usepackage{hyperref}
\usepackage{rotating}  
\usepackage{amsmath}

\newcommand{\todo}[1]{{\bf #1}}
\newcommand{\dd}[1]{\ensuremath{{\rm d}#1}}
\newcommand{\mat}[1]{\ensuremath{{\bf #1}}}
\DeclareMathOperator{\sech}{sech}

\newif\ifanswers

\NewEnviron{answer}{\ifanswers\color{blue}\expandafter\BODY\fi}

\newenvironment{ditemize}
{ \begin{list}{}{%
\setlength{\topsep}{0pt}% 
\setlength{\partopsep}{3pt}% 
\setlength{\itemsep}{1pt}\setlength{\parsep}{1pt}% 
\setlength{\itemindent}{0pt}\setlength{\listparindent}{12pt}%
\setlength{\leftmargin}{24pt}\setlength{\rightmargin}{0in}%
\setlength{\labelsep}{6pt}\setlength{\labelwidth}{6pt}%
\renewcommand{\makelabel}{\makebox[\labelwidth][l]{$\bullet$\hspace{\fill}}}}}
{\end{list}}


\begin{document}

\title{\bf Computational Physics / PHYS-GA 2000 / Problem Set \#6
\\ Due October 31, 2023 }

You {\it must} label all axes of all plots, including giving the {\it
  units}!!

\begin{enumerate}
\item Exercise 5.15 in Newman.
\item This problem will explore interpolation a little so you have
  some experience with it. We will interpolate values of the $\sin()$
  function. In each part below, you will interpolate $\sin()$ from a
  grid of known values at $N$ equally spaced points for $x$ between
  $0$ and $10\pi$ (inclusive).
\begin{enumerate}
\item First, use linear interpolation, writing this code
  yourself. Test your code for $N=20$, $N=40$, $N=80$, and
  $N=160$. Quantify the rms residuals of the interpolation relative to
  $\sin()$ within the range of the grid, as a function of $N$.
\item Second, go ahead and utilize the {\tt interp1d} class in
  {\tt scipy.interpolate} to interpolate. Test the {\tt slinear}, {\tt
    quadratic}, and {\tt cubic} methods in the same way as above.
\item Third, add a little bit of noise to the values of $\sin()$ that
  you interpolate between; use Gaussian noise with a standard
  deviation of 0.1. Show some examples of how the interpolation
  behaves.
\end{enumerate}
  
\end{enumerate}

\end{document}
