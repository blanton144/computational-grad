\documentclass[11pt, preprint]{aastex}

\usepackage{environ}
\usepackage{xcolor}
\usepackage{hyperref}
\usepackage{rotating}  
\usepackage{amsmath}

\newcommand{\todo}[1]{{\bf #1}}
\newcommand{\dd}[1]{\ensuremath{{\rm d}#1}}
\newcommand{\mat}[1]{\ensuremath{{\bf #1}}}
\DeclareMathOperator{\sech}{sech}

\newif\ifanswers

\NewEnviron{answer}{\ifanswers\color{blue}\expandafter\BODY\fi}

\newenvironment{ditemize}
{ \begin{list}{}{%
\setlength{\topsep}{0pt}% 
\setlength{\partopsep}{3pt}% 
\setlength{\itemsep}{1pt}\setlength{\parsep}{1pt}% 
\setlength{\itemindent}{0pt}\setlength{\listparindent}{12pt}%
\setlength{\leftmargin}{24pt}\setlength{\rightmargin}{0in}%
\setlength{\labelsep}{6pt}\setlength{\labelwidth}{6pt}%
\renewcommand{\makelabel}{\makebox[\labelwidth][l]{$\bullet$\hspace{\fill}}}}}
{\end{list}}


\begin{document}

\title{\bf Computational Physics Project / Stellar Structure}

This project involves a simplified calculation of stellar structure,
under spherical symmetry.

The general equations governing stellar structure in spherical
symmetry are mass conservation, the hydrostatic equation, energy
conservation, and energy transport. It turns out to be useful to
consider these equations not as a function of radius, but as a
function of enclosed mass within the radius.

A good introduction to the background behind the equations can be
found in
\href{http://www.ast.cam.ac.uk/~pettini/STARS/Lecture09.pdf}{\color{blue}
  these lecture notes}. To summarize, for a star with energy transport
dominated by radiative diffusion:
\begin{eqnarray}
\frac{\dd{r}}{\dd{m}} &=& \frac{1}{4\pi r^2 \rho} \cr
\frac{\dd{P}}{\dd{m}} &=& -G \frac{m}{4\pi r^4} \cr
\frac{\dd{L}}{\dd{m}} &=& \mathcal{E} \cr
\frac{\dd{T}}{\dd{m}} &=& - \frac{3}{16 \sigma } \frac{\kappa}{T^3}
\frac{L}{(4\pi r^2)^2}
\end{eqnarray}
where $r$ is the radius, $m$ is the mass enclosed within $r$, $L$ is
the luminosity generated within $r$, $P$ is the pressure, $T$ is the
temperature, and $\rho$ is the density. $\mathcal{E}$ is the energy
generated per unit mass and $\kappa$ is the opacity of the
gas. $\sigma$ and $c$ are the Stefan-Boltzman constant and the speed
of light.

If we want to solve these equations, we need four boundary conditions;
these are that $r=0$ and $L=0$ at $m=0$, and that $P=0$ and $T=0$ at
$m=M_\ast$.

We also need to know how $\mathcal{E}$ and $\kappa$ behave. Generally:
\begin{equation}
\mathcal{E} = \mathcal{E}_0 \rho^\alpha  T^\beta
\end{equation}
where here we will take $\alpha=1$ and $\beta = 4$. 

The opacity $\kappa$ can be approximated as :
\begin{equation}
\kappa = \kappa_0 \rho T^{-3.5}
\end{equation}

Finally, we need the equation of state, which in the cases we consider
here can be approximated as :
\begin{equation}
P = \frac{\rho kT}{\mu m_p}
\end{equation}
where $\mu$ is the mean molecular weight in units of proton masses,
and is 0.6.

The parameters of the system that are not physical constants are
$M_\ast$, $\kappa_0$,  and $\mathcal{E}_0$.

\section{Prep work}

\begin{itemize}
\item Recast the above equations into a form with the minimum number
  of independent, unitless parameters.  You should perform your
  numerical analysis in these variables, and then scale your solution
  after the fact to specific physical solutions.
\item {\bf polytrope solution}
\end{itemize}

\section{Creating the integrator}

\begin{itemize}
\item Build an integrator for the ODE system working from the center
  outwards, given some initial conditions at the center.
\item Wrap the integration in a multidimensional minimization routine
  that adjusts the central conditions of $T$ and $P$ to fit the outer
  boundary conditions.
\end{itemize}

\section{Testing the integrator}

\begin{itemize}
\item Plot $\rho$, $P$, $T$, $L$ and $\mathcal{E}$ as a function of
  enclosed radius for some choice of $M_\ast$.
\item Given the Solar values of mass and luminosity, can you determine
  some combination of $\kappa_0$ and $\mathcal{E}_0$?
\end{itemize}

\end{document}
