\documentclass[11pt, preprint]{aastex}

\usepackage{environ}
\usepackage{xcolor}
\usepackage{hyperref}
\usepackage{rotating}  
\usepackage{amsmath}

\newcommand{\todo}[1]{{\bf #1}}
\newcommand{\dd}[1]{\ensuremath{{\rm d}#1}}
\newcommand{\mat}[1]{\ensuremath{{\bf #1}}}
\DeclareMathOperator{\sech}{sech}

\newif\ifanswers

\NewEnviron{answer}{\ifanswers\color{blue}\expandafter\BODY\fi}

\newenvironment{ditemize}
{ \begin{list}{}{%
\setlength{\topsep}{0pt}% 
\setlength{\partopsep}{3pt}% 
\setlength{\itemsep}{1pt}\setlength{\parsep}{1pt}% 
\setlength{\itemindent}{0pt}\setlength{\listparindent}{12pt}%
\setlength{\leftmargin}{24pt}\setlength{\rightmargin}{0in}%
\setlength{\labelsep}{6pt}\setlength{\labelwidth}{6pt}%
\renewcommand{\makelabel}{\makebox[\labelwidth][l]{$\bullet$\hspace{\fill}}}}}
{\end{list}}


\begin{document}

\title{\bf Computational Physics Project / Gravity on a mesh}

This project focuses on solving Poisson's equation on a uniform mesh.
The problem you will solve here involves a technique known as {\it
  particle-mesh} which is often used in the context of simulating the
formation of structure in the universe and in other contexts.

The version of Poisson's equation we need to solve is:
\begin{equation}
\nabla^2 \phi(\vec{x}) = 4\pi G \rho(\vec{x})
\end{equation}

In order to follow how structure forms due to gravity, the
particle-mesh technique approximates the distribution of matter using
a large set of particles in 3-dimensions. The simulation starts at
some initial time step with some initial conditions for the
particles. The density field is approximated by counting the number of
particles in each cell of a rectangular mesh. This yields a 3D array
containing the density field. Then, Poisson's equation is solved for
the gravitational potential. The acceleration on each particle can
then be calculated as the gradient of the potential. The velocity of
the particle is updated, and its position is updated according to the
velocity. Then the process is repeated with the new (slightly
different) density field. The basics of this method were described
thoroughly in a classic book by Hockney \& Eastwood (1988), {\it
  Computer Simulation Using Particles}.

\section{Rescaling the problem}

First it is important to scale certain factors out of the problem for
convenience: 
\begin{itemize}
\item Define a new unitless set of variables for position, mass,
  velocity, and time, by scaling out the length scale, time scale,
  total mass scale, and $G$. Set the size of the cubic region we are
  considering to unity, and $G=1$, in these new units. There should be
  a single combination of these overall scale values that
  characterizes the expression. You have the freedom to choose to set
  this combination to unity. You should perform your numerical
  analysis in these variables; your numerical solutions can then be
  scaled to different total mass and lengths by keeping this
  combination fixed.
\end{itemize}

\section{Particle positions and density field}

The first thing you will need to do is to define a set of particle
positions and infer a density field. We will use the simplest {\it
  cloud-in-cell} approach to this problem. You should use $32^3$
particles and a $32\times 32\times 32$ grid, so that things run fast.

\begin{itemize}
\item First, you need to distribute a bunch of particles in the 3D
  space to represent a given density field. Write a function to
  distribute particles according to a multi-variate Gaussian with a
  chosen center, chosen semimajor axis $a$, and chosen axis ratios
  ($b/a$ and $c/a$). It is fine if you let the principal axes of the
  Gaussian be aligned with the coordinate system you are working in.
\item Second you need to define a density field based on the
  particles. To do so, imagine each particle is a small cubicle
  ``cloud'' with sides equal to the size of a grid cell. Then the
  contribution of each particle to the density in each cell is just
  the fractional overlap of the particle and the cell. Write a
  function that evaluates the density field and make some plots to
  show that it is working as expected.
\end{itemize}

\section{Solving Poisson's equation}

Then you need to solve Poisson's equation:
\begin{equation}
\nabla^2 \phi(\vec{x}) = 4\pi \rho(\vec{x})
\end{equation}
This is a partial differential equation (PDE), but you do not need to
wait until we learn about PDE solutions in order to solve this
equation. You can use Fourier methods to perform the solution ---
i.e. use the fact that the Fourier transform of the above equation is:
\begin{equation}
k^2 \tilde\phi(\vec{k}) = 4\pi \tilde\rho(\vec{k})
\end{equation}
However! You cannot use precisely this formula with a discrete Fourier
Transform. Instead, there is a discrete version of this formula. 

\begin{itemize}
\item Write the derivation of and implement the discrete version of
  the Fourier solution described above. Test the case of a point
  source at the center of the box. The result is a Green's function
  response of the method to a delta function source. Show that its
  potential is not what you would expect from an isolated delta
  function source, but that it is what you would expect with a
  periodically repeated delta function source.
\item To simulate an isolated distribution of mass, you have to use a
  special trick. First, you need to ``isolate'' the mass distribution
  with a moat of zero density, by expanding the grid size by a factor
  of two in each dimension. Second, you need to limit the range of the
  Green's function, so that the periodic isolated regions cannot
  influence each other. To do the latter, you need to create the
  Green's function response you want numerically:
  \begin{equation}
    g(\vec{r}) = \frac{1}{r} = \frac{1}{\sqrt{x^2 +y^2 + z^2}}
  \end{equation}
  for $|x|<1$, $|y|<1$, $|z|<1$, 0 otherwise, and $g(\vec{0}) =
  1$. Then instead of $k^{-2}$, use the discrete Fourier Transform of
  this Green's function.  Hockney \& Eastwood, around pages 212--213,
  explain this. Implement this method and show that it works as
  expected for a delta function source.
\item Test a spherically symmetric case for the potential against the
  analytic expectation, to estimate the approproximation error.
\item Use your implementation to calculate the potential for several
  cases with different widths and axis ratios of the Gaussian.
\item Conceive of and write at least one and preferably more unit
  tests for this part of the code. 
\end{itemize}

\section{Integrating the equations in time}

Now you have initial positions, the resulting density field, and the
potential resulting from that density field. You can now integrate the
system forward in time. The equations for each particle are:
\begin{eqnarray}
\dot{\vec{v}} &=& -\vec{\nabla}\phi \cr
\dot{\vec{x}} &=& \vec{v} 
\end{eqnarray}

\begin{itemize}
\item Implement the Verlet integrator I describe in the notes for
  ODEs, for the particle-mesh case. Between each integer step,
  update the positions according to the velocities. Between each
  half-integer step, update the velocities according to the gradient
  of the potential at the location of the particle; use a first-order
  method to calculate the gradient. 
\item Motivate a choice for your time step---is there a physically
  motivated choice {\it a priori}? Is there a way to set the time step
  dynamically? 
\item Run the simulation starting from an initial spherical
  distributions of particles at rest. Make some plots showing what the
  particles do over time and how the density field and its radial
  dependence changes. 
\item Test the convergence of the method by running at higher
  resolution and comparing the results results between two runs.
\end{itemize}

\section{Exploring the Physics of Gravitational Collapse}

We can now use the code to explore some basic things about
gravitational collapse. Please note that {\it these} questions are
experimental, in the sense that I have no guarantee about what you
will find using the methods implemented here. As you proceed, feel
free to send me results or questions!

\begin{itemize}  
\item Try initial distributions with different axis ratios. How does
  the collapse differ in these cases?
\item The scalar Virial Theorem (Landau \& Lifshitz, {\it Mechanics},
  \S10) states that in steady state the total potential energy $U$ and
  kinetic energy $K$ are related by $U = -2K$. Plot these two
  quantities against each other as a function of time for some of your
  runs.
\item Can you use the Virial Theorem to predict how the size of the
  initial distribution will change when the distribution starts at
  rest?
\item For an initially spherical distribution, can you assign an
  initial set of velocities that will lead to a nearly no evolution in
  the density? 
\item Use an initially spherical distribution, but assign the
  velocities in such a way that there is net angular momentum. For
  this to have a significant effect the angular momentum per particle
  should be a substantial fraction of the average momentum per
  particle times the characteristic size of the system. What happens
  to the distribution over time?
\end{itemize}

\end{document}
