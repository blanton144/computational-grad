\documentclass[11pt, preprint]{aastex}

\usepackage{environ}
\usepackage{xcolor}
\usepackage{hyperref}
\usepackage{rotating}  
\usepackage{amsmath}

\newcommand{\todo}[1]{{\bf #1}}
\newcommand{\dd}[1]{\ensuremath{{\rm d}#1}}
\newcommand{\mat}[1]{\ensuremath{{\bf #1}}}
\DeclareMathOperator{\sech}{sech}

\newif\ifanswers

\NewEnviron{answer}{\ifanswers\color{blue}\expandafter\BODY\fi}

\newenvironment{ditemize}
{ \begin{list}{}{%
\setlength{\topsep}{0pt}% 
\setlength{\partopsep}{3pt}% 
\setlength{\itemsep}{1pt}\setlength{\parsep}{1pt}% 
\setlength{\itemindent}{0pt}\setlength{\listparindent}{12pt}%
\setlength{\leftmargin}{24pt}\setlength{\rightmargin}{0in}%
\setlength{\labelsep}{6pt}\setlength{\labelwidth}{6pt}%
\renewcommand{\makelabel}{\makebox[\labelwidth][l]{$\bullet$\hspace{\fill}}}}}
{\end{list}}


\begin{document}

\title{\bf Computational Physics / PHYS-UA 210 / Problem Set \#7
\\ Due October 29, 2024 }

You {\it must} label all axes of all plots, including giving the {\it
  units}!!

\begin{enumerate}
\item Exercise 6.16 in Newman. But first, replace $\omega^2$ with the
  correct expression based on $M$, $m$, and $R$ (it is unclear to me
  why this is treated as a separate independent parameter!). Then
  rescale the equation so it only depends on $m' = m/M$ and $r'=r/R$.
  Write the code to take any values of those two parameters; you will
  have to carefully write the initial bracketing code. Either use {\tt
    jax} or an analytic derivative, and use Newton's method. Solve,
  with the same routine but with different inputs, the problem for
  values appropriate to the Moon and the Earth, the Earth and the Sun,
  and for the case of a Jupiter-mass planet orbiting the Sun at the
  distance of the Earth.
\item Implement Brent's 1D minimization method. Test it on this
function: $y= (x -0.3)^2 \exp(x)$. Compare to the {\tt
scipy.optimize.brent} implementation results.
\end{enumerate}



\end{document}
