\title{Arrays}

\section{Why we need arrays of numbers}

Last time we discussed data types, in the context of single numbers.
However, if we are bothering to perform a computation with a computer,
we probably have more than one number to deal with. For example, we
might have a million numbers, which we want to either apply the same
computation to, or combine in some way. 

Thus, all computing programming languages have a way to store arrays
of numbers. Often these can be thought of as a one- or
more-dimensional matrix. In most cases, the array is implemented as a
series of memory addresses containing numbers ordered
sequentially. E.g. a section of memory containing an array of 4-byte
floats might look like:

\begin{center}
  \begin{tabular}{ | c | c | c | c | c |}
    \hline
    Address & $+0$ & $+1$ & $+2$ & $+3$ \\ \hline
    0 & 00010001 & 01010101 & 00101010 & 11111100 \cr
    4 & 10111001 & 11000001 & 11011010 & 11010100 \cr
    8 & 01010011 & 10001011 & 01001110 & 01110101 \cr
    \ldots & \ldots & \ldots & \ldots & \ldots \cr
    \hline
  \end{tabular}
\end{center}

\noindent {\bf What formula tells you the address of the $i$th element
  of an array?}

\begin{answer}
To find the address of the $i$th element of an array (``{\tt
  a[i]}'') you can go to address:
\begin{equation}
  {\tt a[i]} = {\tt a[0]} + N_{\rm byte} \times i
\end{equation}
where $N_{\rm byte}$ is the number of bytes in each element. Obviously
this requires all the elements in the array to be the same size.

I have assumed here ``0-indexed'' arrays --- i.e. the first element is
{\tt a[0]}. This is true of many languages, including C and
Python. However, Fortran and some other languages use ``1-indexed''
arrays. It is important to know which indexing you should be using!
\end{answer}

\section{Multidimensional arrays}

Virtually all computers have one-dimensional addressing of memory. So
in order to express a two-dimensional array, the computer reorders
them one-dimensionally, e.g. with row 0, then row 1, etc.

\noindent {\bf What formula tells you the address of the $(i,j)$th
  element of an array?}

\begin{answer}
If the
shape of the array is $(P, Q)$, then the address of element $(i,j)$,
in Python denoted {\tt{a[i ,j]}} becomes:
\begin{equation}
  {\tt a[i, j]} = {\tt a[0]} + N_{\rm byte} \times (i \times Q + j)
\end{equation}

Another thing that tends to differ among languages is how what the
ordering of $(i, j)$ implies about which numbers are next to each
other in memory; that is, whether I need to write $i\times Q +j$ or $i
+ j\times P$ in the above equation.

There is a lot of confusing documentation out there about ``rows'' and
``columns.'' In 2D arrays the ``row'' is indicated by the first index
the ``column'' is indicated by the second index. ``Row-major'' order
is the order shown in the equation above --- the bigger steps in
memory are the rows.  ``Column-major'' is the opposite.

For C, the convention is row-major, as shown above. The Fortran
convention is column-major.

Now, let us consider NumPy. There is a default behavior of NumPy which
is C-like (row-major, like the example above). However, you can also
set NumPy to act differently. As I show in the notebook, you can
actually get it to show you the ``strides'' each index implies in
memory space.
\end{answer}

Why am I emphasizing the location of the array elements in memory? It
is because when you are dealing with large arrays in memory (and on
disk) it really matters where the data is located. Your computers CPU
needs to get information from its memory (RAM). The RAM is typically
about a few Gbytes, but accessing it is slow; you don't want to go
fetch every byte of information individually. Instead, what your
computer does is gets big sections of memory from RAM simultaneously
and put it into a cache near the CPU, where the access time is
fast. Usually there are multiple layers of cache, with increasingly
large sizes but slower access, up to 100 Mb or so. There is all sorts
of strategy associated with this, done at the level of the compiler,
optimizer, and in the CPU. We won't cover these issues now.

But the order in which you sort your data on the computer is something
you typically control at a high level, that really impacts this
process. In the examples I show in the notebook, you can see that the
order matters quite a bit. Again, in many languages an optimizer will
catch this and fix it (when it can be sure that it won't change the
answer) but you need to be aware of this.

\noindent {\bf How do you express 3- or more-dimensional arrays?}

\begin{answer}
This is just a generalization of the 2-dimensional case:
\begin{equation}
  {\tt a[i, j, k]} = {\tt a[0]} + N_{\rm byte} \times (i \times (P
  \times Q) + j \times Q + k)
\end{equation}
for an array of shape $(P, Q, R)$.
\end{answer}

\section{NumPy arrays}

Now let's look closer at NumPy specifically. The first thing to note
is that NumPy arrays are completely different than Python lists. They
use similar syntax but they are very different objects. Lists are much
more flexible and correspondingly less efficient for numerical
computation.

The NumPy array object class is called {\tt ndarray}. It is quite
flexible, but does assume every element of the array has the same {\tt
  dtype}. 

One of the nice things about NumPy arrays are that they allow
functions to act on them as whole objects, or on some range of their
elements. This makes programming simpler and also can allow the
implementation to run more efficiently, as we will see. I'll note that
Python is not the 100\% best programming language for this sort of
computation; the new language Julia is probably better, and Fortran 90
is excellent in this respect. 

In the notebook I demonstrate the various ways in which arrays can be
accessed and used. A more full demonstration can be found in the
\href{https://github.com/jakevdp/PythonDataScienceHandbook/tree/de0cc6bd317012d50ab3dd06e3cf4e256de1973f/notebooks}{PDSH,
  Chapter 2} (this is true of the rest of the discussion here).

There are also a set of ``ufuncs'' for NumPy arrays. These are
generally much faster to apply than by looping through each array
element with a {\tt for} loop. What happens in a {\tt for} loop is
that at every iteration, the machine checks the types of variables it
is operating on and performs other bureaucratic tasks, which if it
{\it knew} it was going to do over and over again, it would only do
once. Within the ufunc it can assume that, and save a lot of time. It
also allows the computer to vectorize at a lower level as well
(pipelining multistep operations so it is working on multiple array
elements at once). 

We demonstrate several other methods as well:
\begin{ditemize}
\item {\tt min()}, {\tt max()}
\item {\tt sum()}
\item Comparisons ($<$, $>$, \ldots)
\item Indexing
\item Sorting
\end{ditemize}

\section{Quantifying the scaling of algorithmic cost}

It is important when you are considering any implementation of a
computational scheme to quantify how expensive that computation will
be. Usually the most important factor is how the algorithm {\it
  scales} with problem size. This scaling is usually denoted by the
$\mathcal{O}()$ notation. 

Sorting is a good example. Above, we just called a ``sort'' function
that did all the hard work in a clever way, since sorting is an
ancient algorithmic problem in computer science. However, there are 
also dumbs way to sort.

One way to sort that you would probably think of first is called
``insertion sort.'' It is basically how you sort a hand of cards. You
start with the first two elements of the array and make sure they are
sorted properly. Then, you check the third element, and insert it in
the right place by scanning the already sorted elements (the first
two). Then you just repeat this. Even if you are clever about dealing
with the insertion steps, this means that for each of the $N$ elements
of the arrays, you need to scan on average $\sim N/2$ other
elements. This means the number of comparisons in the sort is $\sim
N^2/2$. This is what we refer to as an $\mathcal{O}(N^2)$ algorithm.

NumPy's {\tt sort()} method doesn't give you insertion sort as an
option. It allows a few others, like {\tt quicksort}, that have much
better scaling. {\tt quicksort} goes as follows:
\begin{enumerate}
\item Choose one element in the array as a pivot.
\item Partition the array so that values less than the pivot are on
  one side, and values greater than are on the other (this step is
  $\mathcal{O}(N)$).
\item Recursively apply (1) and (2) to the subarrays on the left and
  right of the pivot.
\end{enumerate}
Implementing this efficiently is still tricky, but notice that the way
the problem is structured is that you need to perform an
$\mathcal{O}(N)$ algorithm around $\log_2N$ times. So it scales as
$\mathcal{O}(N\log N)$.

For large $N$ there is a very large difference in cost!!

{\tt quicksort} was invented about 60 years ago. It is important to
remember as you do computational physics that there are a {\it lot} of
tricks computer scientists have learned over the years. You should not
be inventing them all yourself.
