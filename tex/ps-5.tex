\documentclass[11pt, preprint]{aastex}

\usepackage{environ}
\usepackage{xcolor}
\usepackage{hyperref}
\usepackage{rotating}  
\usepackage{amsmath}

\newcommand{\todo}[1]{{\bf #1}}
\newcommand{\dd}[1]{\ensuremath{{\rm d}#1}}
\newcommand{\mat}[1]{\ensuremath{{\bf #1}}}
\DeclareMathOperator{\sech}{sech}

\newif\ifanswers

\NewEnviron{answer}{\ifanswers\color{blue}\expandafter\BODY\fi}

\newenvironment{ditemize}
{ \begin{list}{}{%
\setlength{\topsep}{0pt}% 
\setlength{\partopsep}{3pt}% 
\setlength{\itemsep}{1pt}\setlength{\parsep}{1pt}% 
\setlength{\itemindent}{0pt}\setlength{\listparindent}{12pt}%
\setlength{\leftmargin}{24pt}\setlength{\rightmargin}{0in}%
\setlength{\labelsep}{6pt}\setlength{\labelwidth}{6pt}%
\renewcommand{\makelabel}{\makebox[\labelwidth][l]{$\bullet$\hspace{\fill}}}}}
{\end{list}}


\begin{document}

\title{\bf Computational Physics / PHYS-UA 210 / Problem Set \#5
\\ Due October 17, 2019 }

You {\it must} label all axes of all plots, including giving the {\it
  units}!!

\begin{enumerate}
  \item Exercise 5.15 in Newman. Additionally, use {\tt jax} to
    perform the autodiff version of the derivative to verify it works
    as advertised.
  \item Exercise 5.17 in Newman.
  \item This problem demonstrates an application of linear algebra to
    signal analysis. Download a ``signal'' as a function of time from
    \href{https://github.com/blanton144/computational-grad/blob/main/data/signal.dat}{this
      file}. Assume that all the measurements have the same
    uncertainty, with a standard deviation of 2.0 in the signal units.
    \begin{enumerate}
      \item Plot the data.
      \item Use the SVD technique to find the best third-order
        polynomial fit in time to the signal. Pay attention to the
        scaling of the independent variable (time).
      \item Calculate the residuals of the data with respect to your
        model. Argue that this is not a good explanation of the data
        given what you know about the measurement uncertainties.
      \item Try a much higher order polynomial. Is there any
        reasonable polynomial you can fit that {\it is} a good
        explanation of the data? Define ``reasonable polynomial'' as
        whether the design matrix has a viable condition number.
      \item Try fitting a set of $\sin$ and $\cos$ functions plus a
        zero-point offset. As a Fourier series does, use a harmonic
        sequence with increasing frequency, starting with a period
        equal to half of the time span covered. Does this model do a
        ``good job'' explaining the data? Are you able to determine a
        typical periodicity in the data? You may have noticed a
        periodicity from the first plot.
    \end{enumerate}
    This last fit is a version of the {\it Lomb-Scargle} technique for
    detecting variability in unevenly sampled data, which is designed
    to be a very close approximation to fitting with a set of Fourier
    modes. Implementing the method the way we do here (explicitly
    decomposing the design matrix) is not the usual method, since it
    is slower than other techniques, but it is the simplest to
    implement.
\end{enumerate}

\end{document}
