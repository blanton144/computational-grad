\documentclass[11pt, preprint]{aastex}

\usepackage{environ}
\usepackage{xcolor}
\usepackage{hyperref}
\usepackage{rotating}  
\usepackage{amsmath}

\newcommand{\todo}[1]{{\bf #1}}
\newcommand{\dd}[1]{\ensuremath{{\rm d}#1}}
\newcommand{\mat}[1]{\ensuremath{{\bf #1}}}
\DeclareMathOperator{\sech}{sech}

\newif\ifanswers

\NewEnviron{answer}{\ifanswers\color{blue}\expandafter\BODY\fi}

\newenvironment{ditemize}
{ \begin{list}{}{%
\setlength{\topsep}{0pt}% 
\setlength{\partopsep}{3pt}% 
\setlength{\itemsep}{1pt}\setlength{\parsep}{1pt}% 
\setlength{\itemindent}{0pt}\setlength{\listparindent}{12pt}%
\setlength{\leftmargin}{24pt}\setlength{\rightmargin}{0in}%
\setlength{\labelsep}{6pt}\setlength{\labelwidth}{6pt}%
\renewcommand{\makelabel}{\makebox[\labelwidth][l]{$\bullet$\hspace{\fill}}}}}
{\end{list}}


\begin{document}

\title{\bf Computational Physics Project / Telescope Diffraction Limit}

This project involves a calculation of the diffraction pattern due to
the finite aperture of a telescope.

If a plane wave enters a well-focused telescope along its axis of
symmetry (on-axis), then geometric optics tells us that all of the
light will be focused at a single point. However, because light is a
wave, in fact it will form a finite-sized image at the focus. The size
and shape of the image is related to the aperture of the telescope.

Specifically, the image formed (the point spread function) can be
shown to be the squared amplitude of the Fourier transform of the
complex pupil function:
\begin{equation}
P(r, \phi) \exp\left(-ik W(r,\phi)\right)
\end{equation}
where $P$ is a function of position on the aperture, in a coordinate
system centered on the aperture axis of symmetry, and $W$ is a phase
shift (induced, for example, by imperfections in the shape of the
mirror). $k= 2\pi/\lambda$ is the wavenumber of the light wave.

In the simplest case, $P(r, \phi)$ is just $1$ within some radius
$D/2$, and $0$ outside.  In this case, the image formed is:
\begin{equation}
I = C\left(\frac{J_1(x)}{x}\right)^2
\end{equation}
where $J_1$ is the first-order Bessel function, and $x=\pi
D\theta/\lambda$, and $\theta$ is a radial coordinate in the focal
plane related to the input angle of light into the telescope (we call
it $\theta$ because it corresponds to an angle on the sky). This is
known as the Airy pattern.

\section{Prep work}

\begin{itemize}
\item $J_1$ can be calculated as:
  \begin{equation}
   J_1 = \frac{1}{\pi} \int_0^\pi \dd\theta\, \cos\left(\theta -
   x\sin\theta\right)
  \end{equation}
  You should in general use numerical libraries to calculate it.  But
  this one time use Simpson's rule to check the NumPy implementation
  of the calculation. Note that NumPy (and other libraries) use a very
  accurate analytic approximation to evaluate this function, not an
  explicit integral.
\item Many reflective telescopes have a Cassegrain design
  (\href{https://en.wikipedia.org/wiki/Cassegrain_reflector}{\color{blue}
  see Wikipedia}) which causes the aperture to be a donut, with a hole
  in the middle. Can you write analytically (in terms of $J_1$) what
  you expect the PSF to be? Plot the result.
\item We will want to simulate small imperfections in the phases from
  different causes. To do so it will be useful to be able to create
  {\it Gaussian random fields}. These are random fields whose values
  have Gaussian distributions, and have random ``phases.''
  Specifically, they can be created by choosing independent Gaussian
  random values for Fourier modes in $\vec{k}$-space, and Fourier
  transforming back to configuration space. The random distributions
  have mean zero and a variance that is a function of the scalar $k$,
  called $P(k)$ or the ``power spectrum.'' First, create a function
  that returns a set of amplitudes $a(\vec{k})$ for each mode. The
  phase differences will be real, but the Fourier amplitudes will not
  be. You must set the amplitudes with the right symmetries to
  guarantee that their Fourier transform will be real!
\item Now create a function that will use the function from the
  previous step plus an FFT to produce a random
  field with some user-specified $P(k)$. You may consult
  \href{http://andrewwalker.github.io/statefultransitions/post/gaussian-fields/}{\color{blue}
    the example code here}. Try using
  \begin{equation}
    P(k) \propto k^{n} \exp\left(- k^2 / k_c^2\right)
  \end{equation}
    for several values of $n<0$ and $k_c$. Plot the random fields
    produced and comment on the differences.
\end{itemize}

\section{Calculating PSFs}

\begin{itemize}
\item Perform a numerical FFT of a simple circular aperture, in order
  to verify the above result in detail. Show how the size of the PSF
  varies with aperture size.
\item For a Cassegraine telescope, to hold the secondary mirror in
  place there need to be some structural elements.  Often these take
  the form of several struts holding it in place, which cause
  obscuration of the aperture (basically, small lines with $P=0$
  aligned radially). Look up some images of stars on the web from the
  Hubble Space Telescope, guess how many such struts it has, and try
  to reproduce the diffraction spikes (roughly) using your model. Do
  you notice anything about the patterns of color in the diffraction
  spikes that you can explain with your model?
\item Small imperfections in the mirror cause phase shifts. Simulate
  such shifts in your model by creating a Gaussian random field for
  the phase offsets $W$ (which is in the same units as $\lambda$) in
  each pixel. Use a $P(k)$ with power at large $k$. Under what
  conditions do you expect these shifts to become a major problem? Can
  you verify that?
\item For a ground-based telescope, the atmosphere causes the light
  coming into the telescope to not be a plane wave. Instead it is
  ``wrinkled'' with a coherence length of about 20 cm. Create a
  Gaussian random field whose power spectrum cuts off on smaller
  scales, and above those scales has fluctuations large compared to
  the wavelength of the light. The random field should look ``smooth''
  on small scales.  How do these errors affect the point spread
  function?
\end{itemize}


\end{document}
