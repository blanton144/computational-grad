\documentclass[11pt, preprint]{aastex}

\include{computational_defs}

\begin{document}

\title{\bf Computational Physics / PHYS-GA 2000 / Problem Set \#2
\\ Due September 17, 2024 }

You {\it must} label all axes of any plots, including giving the {\it
  units}!!

\begin{enumerate}

  \item Figure out how NumPy's 32-bit floating point representation
    (which is the IEEE standard) represents the number $100.98763$ in
    bits. This 32-bit representation in fact corresponds to a slightly
    different number. Calculate what that 32-bit number is and how
    much the actual number differs from its 32-bit floating point
    representation

  \item Demonstrate using examples that 32-bit is less precise and
    smaller dynamic range than 64-bit. First, find approximately the
    smallest value that you can add to 1, and get an answer that is
    different than 1, in both 32-bit ({\tt np.float32}) and 64-bit
    ({\tt np.float64}) precision. Second, approximately find the
    minimum and maximum (positive) numbers that these two types can
    represent without an underflow or overflow.

  \item Exercise 2.9 of Newman. Note that the physical constants drop
    out so you do not need to worry about them (whenever possible you
    should seek to remove physical constants from the innards of your
    computations!).  Write two versions of the code, one which uses a
    {\tt for} loop and one which does not.  Use the {\tt timeit}
    module to determine which is faster.

  \item Exercise 3.7 of Newman.  Note that you can use a NumPy array
    to perform the iterations for each value of $c$ all at once, which
    will be much faster than using a {\tt for} loop over $c$.

  \item Exercise 4.2 of Newman. For Part (c), put your solver into a
    {\it module} that can be imported. Name the module {\tt
      quadratic}, and name your function within it {\tt
      quadratic}. Create a file called {\tt test\_quadratic.py} with
    the contents shown below, and ensure that the call {\tt pytest
      test\_quadratic.py} returns no errors. This is a simple example
    of a unit test.
\begin{verbatim}
import pytest

import numpy as np
import quadratic

def test_quadratic():
    # Check the case from the problem
    x1, x2 = quadratic.quadratic(a=0.001, b=1000., c=0.001)
    assert (np.abs(x1 - (- 1.e-6)) < 1.e-10)
    assert (np.abs(x2 - (- 0.999999999999e+6)) < 1.e-10)

    # Check a related case to the problem
    x1, x2 = quadratic.quadratic(a=0.001, b=-1000., c=0.001)
    assert (np.abs(x1 - (0.999999999999e+6)) < 1.e-10)
    assert (np.abs(x2 - (1.e-6)) < 1.e-10)

    # Check a simpler case (note it requires the + solution first)
    x1, x2 = quadratic.quadratic(a=1., b=8., c=12.)
    assert (np.abs(x1 - (- 2.)) < 1.e-10)
    assert (np.abs(x2 - (- 6)) < 1.e-10)
\end{verbatim}

\end{enumerate}

\end{document}
