\documentclass[11pt, preprint]{aastex}

\usepackage{environ}
\usepackage{xcolor}
\usepackage{hyperref}
\usepackage{rotating}  
\usepackage{amsmath}

\newcommand{\todo}[1]{{\bf #1}}
\newcommand{\dd}[1]{\ensuremath{{\rm d}#1}}
\newcommand{\mat}[1]{\ensuremath{{\bf #1}}}
\DeclareMathOperator{\sech}{sech}

\newif\ifanswers

\NewEnviron{answer}{\ifanswers\color{blue}\expandafter\BODY\fi}

\newenvironment{ditemize}
{ \begin{list}{}{%
\setlength{\topsep}{0pt}% 
\setlength{\partopsep}{3pt}% 
\setlength{\itemsep}{1pt}\setlength{\parsep}{1pt}% 
\setlength{\itemindent}{0pt}\setlength{\listparindent}{12pt}%
\setlength{\leftmargin}{24pt}\setlength{\rightmargin}{0in}%
\setlength{\labelsep}{6pt}\setlength{\labelwidth}{6pt}%
\renewcommand{\makelabel}{\makebox[\labelwidth][l]{$\bullet$\hspace{\fill}}}}}
{\end{list}}


\begin{document}

\title{\bf Computational Physics / PHYS-GA 2000 / Problem Set \#2
\\ Due September 19, 2023 }

You {\it must} label all axes of any plots, including giving the {\it
  units}!!

\begin{enumerate}

  \item Figure out how NumPy's 32-bit floating point representation
    (which is the IEEE standard) represents the number $100.98763$ in
    bits. By how much does the actual number differ from its 32-bit
    floating point representation?

  \item Exercise 2.9 of Newman. Note that the physical constants drop
    out so you do not need to worry about them (whenever possible you
    should seek to remove physical constants from the innards of your
    computations!).  Write two versions of the code, one which uses a
    {\tt for} loop and one which does not.  Use {\tt \%timeit} to
    determine which is faster.

  \item Exercise 3.7 of Newman.  Note that you can use a NumPy array
    to perform the iterations for each value of $c$ all at once, which
    will be much faster than using a {\tt for} loop over $c$.

  \item Exercise 4.2 of Newman.

\end{enumerate}

\end{document}
