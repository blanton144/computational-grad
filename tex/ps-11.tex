\documentclass[11pt, preprint]{aastex}

\usepackage{environ}
\usepackage{xcolor}
\usepackage{hyperref}
\usepackage{rotating}  
\usepackage{amsmath}

\newcommand{\todo}[1]{{\bf #1}}
\newcommand{\dd}[1]{\ensuremath{{\rm d}#1}}
\newcommand{\mat}[1]{\ensuremath{{\bf #1}}}
\DeclareMathOperator{\sech}{sech}

\newif\ifanswers

\NewEnviron{answer}{\ifanswers\color{blue}\expandafter\BODY\fi}

\newenvironment{ditemize}
{ \begin{list}{}{%
\setlength{\topsep}{0pt}% 
\setlength{\partopsep}{3pt}% 
\setlength{\itemsep}{1pt}\setlength{\parsep}{1pt}% 
\setlength{\itemindent}{0pt}\setlength{\listparindent}{12pt}%
\setlength{\leftmargin}{24pt}\setlength{\rightmargin}{0in}%
\setlength{\labelsep}{6pt}\setlength{\labelwidth}{6pt}%
\renewcommand{\makelabel}{\makebox[\labelwidth][l]{$\bullet$\hspace{\fill}}}}}
{\end{list}}


\begin{document}

\title{\bf Computational Physics / PHYS-UA 210 / Problem Set \#11
\\ Due November 22, 2019}

You {\it must} label all axes of all plots, including giving the {\it
  units}!!

\begin{enumerate}
  \item Exercise 8.3 of Newman.

\item Write a routine that integrates the equations for projectile
  motion:
  \begin{equation}
    \frac{\dd{^2\vec{x}}}{\dd{t}^2} = - g {\hat x}_1 - \alpha
    \dot{\vec{x}}^2,
  \end{equation}
  where ${\hat x}_1$ is the vertical direction.
  These are appropriate for, say, a golf ball. 
  The initial conditions should be that the object is launched at some
  angle $\theta$ from the horizontal at some initial speed in the
  $x_0$-$x_1$ plane. Integrate until the object hits the ground
  again. Use a Runge-Kutta method
  from {\tt scipy} to solve this problem and write a routine that
  finds where the ball hits the ground again.
\item Use Brent's method (either yours or {\tt scipy}'s) to optimize
  the angle $\theta$ to get the longest distance.

  \item Exercise 8.10 of Newman.
\end{enumerate}

\end{document}
